
\chapter{Citas al proyecto}

\section{Carta de reconocimiento de John Breslin}

El Dr. John Breslin\footnote{\url{http://www.deri.ie/about/team/member/john_breslin/}},
líder del subcluster de software social de Deri Galway\footnote{\url{http://www.deri.ie/}},
el 29 de Noviembre de 2006 ha tenido la amabilidad de dedicar estas alentadoras palabras 
al proyecto:

\begin{quote}
 The SWAML project fulfills a much-needed requirement for the Semantic 
 Web: to be able to refer to semantic versions of email messages and 
 their properties using a resource URI.  I am delighted that the SWAML 
 group have decided to use our SIOC vocabulary for their work, especially 
 now as SIOC is beginning to achieve traction in terms of both metadata 
 creation and applications that can make use of this metadata.  By 
 reusing the SIOC vocabulary for describing online discussions, SWAML 
 allows users of SIOC to refer to email messages from other discussions 
 taking place on forums, blogs, etc., so that distributed conversations 
 can occur across these discussion media.  Also, by providing email 
 messages in SIOC format, SWAML are providing a rich source of data, 
 namely mailing lists, for use in SIOC applications.  I am also happy to 
 see the SWAML creators developing their own applications that will work 
 with SIOC data - the Buxon sioc:Forum visor is a great example of a 
 program using SIOC message data that can come from one or many sources 
 (e.g. from a virtual forum or container of posts from multiple sites 
 and systems).
\end{quote}

\newpage

\section{Citas en blogs}

El modelo de desarrollo abierto y colaborativo ha conseguido que el proyecto haya 
entrado de manera natural en la comunidad científica que trabaja en el campo de la
Web Semántica. Esto ha hecho que dicha comunidad haya seguido con interés los avances 
en el proyecto, coleccionando un puñado de citas que siempre son de agradecer:

\subsection*{SIOC News\footnote{\url{http://apassant.net/blog/post/2006/10/01/117-sioc-news}}}

Por Alexandre Passant el domingo 1 de Octubre de 2006, día previo al lanzamiento 
de la primera versión de SWAML:

\begin{quote}
 (...) Wikier mentionned on \#sioc that SWAML, a project he's involved in to translate mailing 
 lists in RDF, will use SIOC. (...)
\end{quote}

\subsection*{State of the SIOC-o-sphere\footnote{\url{http://www.johnbreslin.com/blog/2006/11/07/state-of-the-sioc-o-sphere-number-3/}}}

Por John Breslin el martes 7 de Noviembre de 2006, después de la publicación de Buxon:

\begin{quote}
 (...) SWAML, the Semantic Web Archive of Mailing Lists, is now using SIOC as its base 
 ontology. Last week, the developers also announced that SWAML now incorporates Buxon, a sioc:Forum 
 visor written in PyGTK (see screenshot). Excellent stuff  (...)
\end{quote}

\subsection*{Buxon visor for sioc:Forum browsing\footnote{\url{http://www.johnbreslin.com/blog/2006/11/08/buxon-visor-for-siocforum-browsing/}}}

Por John Breslin el miércoles 8 de Noviembre de 2006, dando su impresión de Buxon:

\begin{quote}
 I've been testing out the Buxon visor for browsing SIOC forums, created by the SWAML 
 developers and written in PyGTK.

 So far, it works great (with SWAML-generated data). I used an example script packaged 
 with python-libgmail (archive.py) to download an inbox from a GMail account (subscribed 
 to the sioc-dev mailing list) to mbox format, and then ran swaml.py on that mbox to 
 convert it to SIOC RDF. The resulting RDF is here, and I successfully browsed this with 
 Buxon. Great job, SWAML guys!

 This is a nice demonstrator, and it just remains to do the same for a few more 
 SW-related mailing lists...
\end{quote}

\section{Otras citas}

\subsubsection*{nice phone from Patrick\footnote{\url{http://flickr.com/photos/leobard/294518500/in/pool-iswc/}}}

Uldis Bojars comentó\footnote{\url{http://groups.google.com/group/sioc-dev/browse_thread/thread/c9689c4762390396/674fa1a9fedfc1ac}}
en la lista de correo de SIOC-Dev la foto que habían hecho en el
ISWC2006\footnote{\url{http://iswc2006.semanticweb.org/}} con 
PlanetRDF\footnote{\url{http://planetrdf.com/}} visto desde un móvil. Con la casualidad,
como se puede ver en la figura~\ref{fig:planetrdf-mobile}, que el post que en ese momento 
estaba en portada era el de John Breslin hablando de Buxon.

\begin{figure}[ht]
	\centering
	\includegraphics{images/screenshots/planetrdf-mobile.png}
	\caption{PlanetRDF visto desde un móvil con un post hablando de Buxon}
	\label{fig:planetrdf-mobile}
\end{figure}


