
\chapter{Citas}

El modelo de desarrollo abierto y colaborativo del proyecto ha conseguido que el proyecto
haya entrado de manera natural en la comunidad cient�fica que trabaja en el campo de la
Web Sem�ntica.

Esto ha hecho que dicha comunidad haya seguido con inter�s los avances en el proyecto,
coleccionando un pu�ado de citas que siempre son de agradecer:

\section*{SIOC News}

URL: \url{http://apassant.net/blog/post/2006/10/01/117-sioc-news}

Por Alexandre Passant\footnote{\url{http://apassant.net/}} el domingo 1 de Octubre de 2006

Permalink: \url{http://apassant.net/blog/post/2006/10/01/117-sioc-news}

\quote{(...) Wikier mentionned on #sioc that SWAML, a project he's involved in to translate mailing 
lists in RDF, will use SIOC. (...)}

\section*{State of the SIOC-o-sphere (number 3)}

Por John Breslin\footnote{\url{http://www.johnbreslin.com/}} el martes 7 de Noviembre de 2006

Permlink: \url{http://www.johnbreslin.com/blog/2006/11/07/state-of-the-sioc-o-sphere-number-3/}

\quote{(...) SWAML, the Semantic Web Archive of Mailing Lists, is now using SIOC as its base 
ontology. Last week, the developers also announced that SWAML now incorporates Buxon, a sioc:Forum 
visor written in PyGTK (see screenshot). Excellent stuff�  (...)}
