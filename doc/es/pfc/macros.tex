
%título según la portada oficial de PFC's de la EUITIO
\renewcommand{\maketitle}{
    \begin{titlepage}
	\begin{center}
                {%
                    \Large\textrm{\textbf{UNIVERSIDAD DE OVIEDO}}\\
                    \vspace{2cm}
                }
                {%
                    \includegraphics[width=3.5cm]{images/uniovi.png}
                    \hspace{5cm}
                    \includegraphics[width=3cm]{images/euitio.png}\\
                    \vspace{1cm}
                }
                {%
                    \Large\textrm{ESCUELA UNIVERSITARIA DE INGENIERÍA TÉCNICA EN INFORMÁTICA DE OVIEDO}\\
                    \vspace{3cm}
                }
                {%
                    \Large\textrm{\textbf{PROYECTO FIN DE CARRERA}}\\
                    \vspace{3cm}
                }
                {%
                    \Large\uppercase{\textrm{SWAML, publicaci\'on de listas de correo en Web Sem\'antica}}\\
                    \vspace{2cm}
                }
	\end{center}
            {
                \bfseries
                \begin{tabular}{cl}
                \begin{tabular}{|p{5cm}|}
                    \hline
                    \\
                    \\
                    \\
                    {\scriptsize \textbf{VºBº del Director del Proyecto}}\\
                    \hline
                \end{tabular} &
                \begin{tabular}{ll}
                        {\footnotesize\textrm\textbf{DIRECTORES:}} &
                        {\textrm\textbf{Jos\'e Emilio Labra Gayo}} \\ &
                        {\textrm\textbf{Diego Berrueta Mu\~noz}} 
                        \vspace{0.5cm}\\
                        {\footnotesize\textrm\textbf{AUTOR:}} &
                        {\textrm\textbf{Sergio Fern\'andez L\'opez}}
                \end{tabular}
                \end{tabular}
            }
	\pagestyle{empty}
	\cleardoublepage
    \end{titlepage}
}

%cabeceras (con el paquete "fancyhdr")
\headheight 15pt

%macro que me ha chivado Frade, extraido del libro «LaTeX, una imprenta en sus manos», para añadir la bibliografía al indice
\let\OLDthebibliography=\thebibliography
\def\thebibliography#1{\OLDthebibliography{#1}%
	\addcontentsline{toc}{chapter}{\bibname}}

%renombrar las tablas (no funciona)
\renewcommand{\tablename}{Tabla}
\renewcommand{\listtablename}{Indice de tablas}

%colores
\definecolor{darkred}{rgb}{0.5, 0, 0}
\definecolor{violet}{rgb}{1, 0, 1}
\definecolor{green}{rgb}{0.3, 0.95, 0.3}
\definecolor{listinggray}{gray}{0.97}

%listings
\lstset{
	basewidth=0.6em,
	backgroundcolor=\color{listinggray},
	basicstyle=\footnotesize\ttfamily,
	keywordstyle=\bfseries\color{violet},
	stringstyle=\color{blue}\itshape,
	commentstyle=\color{green}\itshape,
	showspaces=false,
	showtabs=false,
	showstringspaces=false,
	frame=trbl,
	extendedchars=true,
	numbers=none,
	aboveskip=0.5cm,
	belowskip=0.5cm,
	xleftmargin=0cm,
	xrightmargin=0cm
}

\lstdefinelanguage{RDF}[]{XML}{%
	morekeywords = {rdf:RDF, rdf:about, rdf:resource, rdf:Description,
			foaf:Person, doap:Project, dc:creator,
			rdfs:label, rdfs:comment, rdfs:domain, 
			rdfs:range, rdfsDatatype, rdfs:subClassOf
	}
}

\lstdefinelanguage{OWL}[]{RDF}{%
	morekeywords = {owl:Ontology, owl:Class, owl:versionInfo, 
			owl:ObjectProperty, owl:inverseOf,
			owl:DatatypeProperty, owl:disjointWith
	}
}

\lstdefinelanguage{SPARQL}{%
	morekeywords = {PREFIX, SELECT, DISTINCT, WHERE
	}
}
