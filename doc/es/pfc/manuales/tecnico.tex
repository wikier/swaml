
\section{Manual técnico}

\subsection*{Obtención de código fuente}

El código fuente de este proyecto se puede obtener por cuatro vías diferentes:

\begin{itemize}
  \item En esta propia documentación, bajo el anexo~\ref{sec:source}.
  \item Del CD que se adjunta al tomo, pegado por la parte interior
	de la portada.
  \item Descargando alguna de las versiones
	publicadas\footnote{\url{http://swaml.berlios.de/files}}.
  \item En el Subversion del proyecto\footnote{\url{http://svn.berlios.de/svnroot/repos/swaml/trunk}} 
	en BerliOS. Puede hacer un \emph{checkout} del repositorio:
	\begin{center}
	 \texttt{svn checkout http://svn.berlios.de/svnroot/repos/swaml/trunk swaml}
	\end{center}
\end{itemize}

\subsection*{Conocimientos}

Antes de abordar el estudio y/o modificación de este proyecto es necesario 
disponga de una serie de conocimientos mínimos acerca de las tecnologías
utilizadas en el mismo.

\begin{itemize}
  \item Debe disponer de conocimientos medios del lenguaje de programación 
	\textbf{Python} en que esta escrito la totalidad del proyecto. 
	Evidentemente se dan por supuestos conocimientos de OOP. En la
	bibliografía podrá encontran multiple documentación que le puede ser
	de utilidad.
  \item Conocer bien como funciona el API de 
	\textbf{RDFLib}\footnote{\url{http://rdflib.net/}}, pues el proyecto
	utiliza intensivamente esta biblioteca..
  \item Al menos disponer de nociones básicas sobre 
	\textbf{RDF}\footnote{\url{http://www.w3.org/RDF/}} (Resource Description 
	Framework) y Web Semántica\footnote{\url{http://www.w3.org/2001/sw/}}.
  \item Tener muy presente la especificación de 
	\textbf{SIOC}\footnote{\url{http://rdfs.org/sioc/spec/}} si se necesita
	modifica alguna de las salidas generadas por el proyecto.
\end{itemize}

\subsection*{Primeros pasos}

FIXME(contar como continuar con el proyecto)
