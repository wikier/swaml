
\section{Manual de usuario}

La aplicación entregada se compone en realidad de cico partes:

\begin{itemize}
 \item SWAML
 \item configWizard
 \item FOAF Enricher
 \item KML Exporter
 \item Buxon
\end{itemize}

Cada uno de estas partes toman la forma de un script Python que puede ser 
invocado mediante su intérprete. La ayuda de cada uno de ello se encuentra
disponible llamandolo con la opción \texttt{--help}, aunque se pasará a 
explicar con más detalle el uso de cada una de estas cinco aplicaciones.

\subsection*{SWAML}

Es la aplicación principal, la que desarrolla el proposito principal del
proyecto. Su funcionalidad se provee por medio del script \texttt{swaml.py}.
Su uso es bien sencillo: como se puede ver en la captura de la 
figura~\ref{fig:swaml} se le invoca acomañado de un único parámetro obligatorio 
que inidica la ruta donde esta la configuración que se le quiere pasar a 
SWAML. Inmediatamente se desemboca todo el proceso sin interacción
alguna con el usuario más que las estadísticas informativas que se 
imprimen al final de cada fase importante del proceso. Si no se
imprime ningún error el proceso habrá concluido satisfactoriamente.

\begin{figure}[H]
	\centering
	\includegraphics[width=10cm]{images/screenshots/swaml.png}
	\caption{SWAML}
	\label{fig:swaml}
\end{figure}

\subsection*{configWizard}

Por medio del script \texttt{congWizard.py} se provee un asistente para
ayudar al usario a crear ficheros de configuración según el formato
que debe recibir SWAML. Tal y como se puede ver en la captura de pantalla
de la figura~\ref{fig:configWizard} el script recibe la ruta destino del 
fichero donde se quiera guardar la configuración. El proceso es sencillo: 
el asistente va pidiendo una serie de parametros al usuario, ofreciendole
un valor por defecto, hasta que haya recopilado toda la información necesaria,
volcandolos inmediatamente a disco con el formato adecuado en la ruta 
indicada por el usuario.

\begin{figure}[H]
	\centering
	\includegraphics[width=14cm]{images/screenshots/configWizard.png}
	\caption{configWizard}
	\label{fig:configWizard}
\end{figure}

Un fichero de ejemplo (que se acompaña con la aplicación) podría ser el
siguiente:

\begin{figure}[H]
\begin{lstlisting}
[SWAML]
title = Example mail list
description = Example description
host = http://example.com/
dir = /var/www/lists/archives/example/
url = http://example.com/lists/archives/example/
mbox = /var/lib/mailman/archives/public/example.mbox
format = YYYY-MMM/postID.rdf
to = example@lists.example.com
kml = yes
foaf = yes
\end{lstlisting}
\caption{Ejemplo de fichero de configuración}
\label{fig:ejemplo-config}
\end{figure}

\subsection*{FOAF Enricher}

Aunque esta funcionalidad se provee en el core de la aplicación principal,
en determinados casos puede ser necesario su uso de manera independiente.
Así el script \texttt{foaf.py} recibe la ruta de un fichero RDF con los
suscriptores de una lista de correo, busca el fichero FOAF de casa uno y
lo enriquece con determinadas propiedades (el propio URI del FOAF, fotografía,
coordenadas geográficas, etc).

\begin{figure}[H]
	\centering
	\includegraphics[width=15cm]{images/screenshots/foaf-enricher.png}
	\caption{FOAF Enricher}
	\label{fig:foaf-enricher}
\end{figure}

\subsection*{KML Exporter}

Al igual que en el caso anterior, la funcionalidad dada por este componente
de manera independiente forma también parte de la aplicación principal. En ese
caso el script \texttt{kml.py} coge como primer parámetro la ruta de un fichero
de suscriptores enriquecido con información geográfica y genera otro fichero
en formato KML posicionando geográficamente los suscriptores.

\begin{figure}[H]
	\centering
	\includegraphics[width=15cm]{images/screenshots/kml-exporter.png}
	\caption{KML Exporter}
	\label{fig:kml-exporter}
\end{figure}

\subsection*{Buxon}

FIXME

\begin{figure}[H]
	\centering
	\includegraphics[width=16cm]{images/screenshots/buxon.png}
	\caption{Buxon}
	\label{fig:buxon}
\end{figure}

