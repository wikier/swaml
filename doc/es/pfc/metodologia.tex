
\chapter{Metodolog�a}

La metodolog�a utilizada para la elaboraci�n de este proyecto ha sido 
\emph{RUP}\footnote{\url{http://www-306.ibm.com/software/awdtools/rup/}}
(Rational Unified Process, Proceso Racional Unificado) de IBM.

La raz�n principal fue la genericidad que brinda para el proceso de desarrollo
software, adaptandose perfectamente a desarrollos basados en programaci�n 
orientada a objetos.

Ayudo a implementar determinadas buenas pr�cticas en Ingenier�a del Software:

\begin{itemize}
  \item Desarrollo iterativo
  \item Administraci�n de requisitos
  \item Uso de arquitectura basada en componentes
  \item Control de cambios
  \item Verificaci�n de la calidad del software
\end{itemize}

Por tanto todo el proceso de desarrollo se dividi� en ciclos, teniendo un 
producto final al final de cada ciclo, dividiendo cada ciclo en fases que 
finalizan con un hito importante.

Los ciclos utilizados fueron:

\begin{itemize}
  \item Inicio: se hizo un plan de fases, identificando los principales 
	casos de uso y riesgos.
  \item Elaboraci�n: se hizo un plan de proyecto, completandose los casos 
	de uso para eliminar los riesgos.
  \item Construcci�n: se concentro en la elaboraci�n de un producto totalmente 
	operativo y eficiente, asicomo en un peque�o manual de usuario.
  \item Transici�n: se implemento el producto en el cliente y se publicaron
	varias versiones funcionales. Como consecuencia de esta publicaci�n
	surgieron nuevos requisitos a ser analizados.
\end{itemize}


