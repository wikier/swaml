
\section{An�lisis}

\subsection{Datos}

\subsubsection{Datos de entrada}

La fuente de informaci�n ser� una lista de correo en formato
mbox\footnote{\url{http://rfc.net/rfc4155.html}}, un formato estandarizado
que utilizan la mayoria de los sistemas de gesti�n de listas de correo
(Mailman\footnote{\url{http://www.gnu.org/software/mailman/}}, 
Majordomo\footnote{\url{http://www.greatcircle.com/majordomo/}},
LISTSERV, Listproc y SmartList entre otros).

El formato mbox no es m�s que un fichero de texto plano en el que se van
almacenando consecutivamente los correos que van llegando a la lista. Se 
almacenan tal cual son enviados a la lista, con su cabeceras originales 
completas y en la codificaci�n del cliente de correo del usuario.

Por tanto es f�cil adivinar dos problemas evidentes de este formato:

\paragraph{Identificacores}

FIXME

Cada correo dispone de un identificador (cabecera \texttt{Message-Id}). Cuando
alguien responde un mensaje, el cliente de correo colocar� en el nuevo una 
cabecera (\texttt{In-Reply-To}) con este ID para referirse explicitamente al 
mensaje que se est� respondiendo.

Este el mecanismo especificado en el RFC\footnote{\url{http://rfc.net/rfc2822.html}} 
para las gesti�n de hilos de conversaci�n por medio de correo electr�nico. Y es
mecanismo que utilizado para representar las conversaciones en forma de �rbol, 
tanto en clientes de correo (Evolution, Thunderbird, Outlook, etc) como en sistemas 
convecionales de publicaci�n de listas de correo.

Dicho identificador tiene una forma similar a \texttt{<3C94C55A3B6A@smtp.isp.com>}.

Pero ese ID no es �nico, sino que es asignado por el propio servidor
SMTP\footnote{\url{http://es.wikipedia.org/wiki/SMTP}} (Simple Mail Transfer Protocol) 
de forma arbitraria a la hora de enviar el correo.

Por tanto no deber�a poder usarse, al menos garantizando un rigurosidad extrema
a la hora de identificar cada uno de los mensajes y sus respuestas.

Podr�a usarse algoritmos eur�sticos con el asunto del mensaje, aunque tampoco nos
garantizar�an una fiabilidad absoluta al poder cambiarse el asunto en cualquier
mensaje del hilo.

Pero si existe una aproximaci�n al problema que consigue una efectividad bastante
alta seg�n se ha podido comprobar. Consiste en asumir que cuando hay una respuesta
a un mensaje, existe una alta probabilidad que se est� respondiendo a �ltimo de los 
mensajes enviados con ID repetido.

\paragraph{Codificaci�n}

FIXME

(cada uno de su padre y de su madre, y almacenados en la codificaci�n del
servidor)

\subsubsection{Datos de salida}

FIXME(HTML, XML, RDF, etc...)

\subsection{Casos de uso}

FIXME

\subsection{Descripci�n de actores}

FIXME

\subsection{Escenarios}

FIXME
