
\subsection{Especificaciones secundarias}

\subsubsection{Requisitos del sistema} 

El software no deberá necesitar de unos requerimientos hardware elevados, 
siendo capaz de ejecutarse en un procesador de como mínimo 300Mhz, con 
un mínimo de 32Mb de memoria RAM. 

Los requisitos concretos (procesador de 32 o 64 bits, sistema operativo, 
etc) vendrán determinados por las inherentes restricciones del entorno de 
ejecución escogido para el proyecto.

\subsubsection{Requisitos de documentación} 

La documentación aportada deberá contener manuales para un completo uso 
del software entregado.

Al menos los siguientes tres documentos:

\begin{itemize}
  \item \textbf{Manual técnico} en el que se recoja toda la información que 
	fuera necesaria si un futuro se desea extender parte o la totalidad 
	del software por parte de personas totalmente ajenas al equipo de 
	desarrollo original.
  \item \textbf{Manual de despliegue} describiendo detalladamente todos los 
	requisitos previos y los pasos concretos que se deben seguir para la 
	correcta instalación del software.
  \item \textbf{Manual de usuario} que recoja ayuda detallada en un lenguaje
	no técnico para la correcta utilización del software.
\end{itemize}

Todos los documentos deberán, además de ser entregados impresos en papel a la
hora de entrega del resto de componentes del proyecto, estar disponibles en
formato imprimible (como por ejemplo PDF) en el sitio web público del proyecto.
