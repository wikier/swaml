
\subsection{Especificación de requisitos}

FIXME

\subsubsection{Datos de entrada}

La fuente de información será una lista de correo en formato
mbox\footnote{\url{http://rfc.net/rfc4155.html}}, un formato estandarizado
que utilizan la mayoria de los sistemas de gestión de listas de correo
(Mailman\footnote{\url{http://www.gnu.org/software/mailman/}}, 
Majordomo\footnote{\url{http://www.greatcircle.com/majordomo/}},
LISTSERV, Listproc y SmartList entre otros).

El formato mbox no es más que un fichero de texto plano en el que se van
almacenando consecutivamente los correos que van llegando a la lista. Se 
almacenan tal cual son enviados a la lista, con su cabeceras originales 
completas y en la codificación del cliente de correo del usuario.

Por tanto es fácil adivinar dos problemas evidentes de este formato:

\paragraph{Identificacores}

FIXME

Cada correo dispone de un identificador (cabecera \texttt{Message-Id}). Cuando
alguien responde un mensaje, el cliente de correo colocará en el nuevo una 
cabecera (\texttt{In-Reply-To}) con este ID para referirse explicitamente al 
mensaje que se está respondiendo.

Este el mecanismo especificado en el RFC\footnote{\url{http://rfc.net/rfc2822.html}} 
para las gestión de hilos de conversación por medio de correo electrónico. Y es
mecanismo que utilizado para representar las conversaciones en forma de árbol, 
tanto en clientes de correo (Evolution, Thunderbird, Outlook, etc) como en sistemas 
convecionales de publicación de listas de correo.

Dicho identificador tiene una forma similar a \texttt{<3C94C55A3B6A@smtp.isp.com>}.

Pero ese ID no es único, sino que es asignado por el propio servidor
SMTP\footnote{\url{http://es.wikipedia.org/wiki/SMTP}} (Simple Mail Transfer Protocol) 
de forma arbitraria a la hora de enviar el correo.

Por tanto no debería poder usarse, al menos garantizando un rigurosidad extrema
a la hora de identificar cada uno de los mensajes y sus respuestas.

Podría usarse algoritmos eurísticos con el asunto del mensaje, aunque tampoco nos
garantizarían una fiabilidad absoluta al poder cambiarse el asunto en cualquier
mensaje del hilo.

Pero si existe una aproximación al problema que consigue una efectividad bastante
alta según se ha podido comprobar. Consiste en asumir que cuando hay una respuesta
a un mensaje, existe una alta probabilidad que se esté respondiendo a último de los 
mensajes enviados con ID repetido.

\paragraph{Codificación}

FIXME

(cada uno de su padre y de su madre, y almacenados en la codificación del
servidor)

\subsubsection{Datos de salida}

FIXME(HTML, XML, RDF, etc...)

\subsubsection{Lenguaje de programación}

El problema planteado requiere de una lenguaje de programación que disponga
de determinadas características:

\begin{itemize}
  \item Fácil despliegue: hay que procurar que SWAML se pueda desplegar en
	todo tipo de máquinas, sin excesivos requisitos ni hardware ni software.
	Es importante que SWAML pueda ser invocado por los distintos programadores
	de tarea de que disponen los sistemas operativos (cron y similares).
  \item API para RDF: que disponga de una madura bliblioteca, a poder ser nativa, 
	para manejar RDF (creación de grafos, parseo desde disco/URI, serializado
	a disco y/o bases de datos, consultas SPARQL, etc).
  \item Biblioteca para ficheros mbox: sería interesante disponer de una biblioteca 
	que abstraiga lo mayor posible al proyecto del manejo de ficheros
	mbox\footnote{\url{http://rfc.net/rfc4155.html}} y mensajes de correo 
	electrónico\footnote{\url{http://rfc.net/rfc2822.html}}.
\end{itemize}

Por tanto el cumplimiento de estas tres necesidades principales era lo primero
a valorar entre todos lenguajes de programación candidatos a convertirse en el
lenguaje utilizado para implementar SWAML. 

Pero tambien se iba a tener en cuenta otras cualidades más generales al problema,
como por ejemplo:

\begin{itemize}
  \item Aspectos concretos de la OOP (object-oriented programming, programación 
	orientada a objetos) que cubra.
  \item Sencillez de desarrollo y posterior estudio del código.
  \item Portabilidad de la solución generada.
  \item Posibilidad de usarse compiladores/intérpretes libres.
  \item FIXME
\end{itemize}

Despues de revisar los lenguajes dsiponibles, fueron varios los candidatos para
someterlos a un estudio más profundo:

\paragraph{Java}: Java\footnote{\url{http://java.sun.com/}}
es un lenguaje de programación, desarrollado por Sun Microsystems, orientado a 
objetos muy popular desde hace varios años. Java no se compila a código nativo, 
sino que una JVM (Java Virtual Machine, máquina virtual de Java) ejecuta el 
bytecode previamente compilado.

En la actualidad se disponen de multitud de implementaciones de la máquina virtual
de Java, desde las propietarias (Sun, IBM, HP, etc) hasta las libres (Harmony, GIJ, 
Kafee, IKVM.NET, etc).

Sobre el problema que nos atañe:

\begin{itemize}
  \item Actualmente las JVM existentes cubren un amplio abanico de arquitecturas y 
	sistemas operativos. Aunque Java esté más pensado para su uso en otro tipo
	de entornos (J2EE por ejemplo), puede invocarse perfectamente en modo en
	linea.
  \item Dispone de forma nativa (desarrollado tambien en Java) de la biblioteca para
	manejar RDF más madura actualmente: Jena\footnote{\url{http://jena.sourceforge.net/}}.
	El famework Jena incluye paquetes para múltiples propósitos dentro de la web
	semántica: API para RDF y OWL, persistencia, serializado y soporte para consultas
	SPARQL.
  \item FIXME
\end{itemize}

Quizás la única (y mayor) pega se encuentre en los términos de licencia de las 
mejores implementaciones de la máquina virtual de Java, lo que complicaria de una
manera importante un futura distribución de SWAML de manera totalmente libre,
pues tendría como dependencias paquetes no libres.


\paragraph{Python}: Python\footnote{\url{http://www.python.org/}} es un lenguaje de 
script extremadamente eficiente. Su uso está muy extendido en todos los sistemas 
Unix actuales (GNU/Linux, familia BSD, Solaris, etc), aunque también está disponible\footnote{\url{http://www.python.org/download/}} para la mayoria de sistemas 
operativos actuales (Windows, MacOS y demás).

Se trata de un lenguaje de script mucho más moderno que otros lenguajes hermanos 
tipo bash o perl. Python va más alla, disponiendo en un lenguaje de script 
interpretado y con tipado dinámico de toda la potencia de los lenguajes orientados 
a objetos más modernos.

Respecto a los tres requisitos que nos interesan:

\begin{itemize}
  \item Al tratarse de un lenguaje de script basta dsiponer de un interprete 
	instalado en el sistema para poderlo ejecutar. Además esta caracteristica
	simplifica enormemente la forma de invocarlo desde un programador de
	tareas.
  \item Existen varias posibilidades para manejar RDF desde Python. Algunas son
	bibliotecas nativas desarrolladas también en Python, y otras están
	disponibles en forma de bindings a bibliotecas desarrolladas en otro 
	lenguaje.
	De todas las posibilidades, quizás RDFLib\footnote{\url{http://rdflib.net/}}
	sea la que se encuentra en un estado de desarrollo más avanzado y maduro 
	(persistencia, serialización, consultas SPARQL, etc).
	Además ofrece la posibilidad de \emph{colocar encima} otras bibliotecas,
	como por ejemplo Sparta\footnote{\url{http://www.mnot.net/sw/sparta/}},
	para utilizar determinados conceptos que no contempla RDFLib.
  \item Python dispone una extensa y completa biblioteca estandar, además de contar
	con multitud de bibliotecas para los más variopintos propósitos. Con modulos 
	como mailbox\footnote{\url{http://docs.python.org/lib/module-mailbox.html}}
	e email\footnote{\url{http://docs.python.org/lib/module-email.html}}, el
	problema de acceso primario a los datos (mailbox unix) que SWAML deberá
	consumir se verá resuelto de manera muy eficiente a un nivel de abstracción
	bastante alto.
\end{itemize}

Además es un lenguaje totalmente libre, desde su especificación hasta varias
de sus implementaciones, incluido el interprete oficial.

Mailman\footnote{\url{http://www.gnu.org/software/mailman/}}, el sistema de gestión
de listas de correo más popular hoy en día, también esta escrito en Python, lo que 
facilitaria en gran medida una supuesta integración de SWAML en Mailman.


\paragraph{C\#}: C\#\footnote{\url{http://msdn2.microsoft.com/en-us/vcsharp/aa336809.aspx}} 
es un lenguaje de programación desarrollado por Microsoft, y posteriormente estandarizado por el ECMA\footnote{\url{http://www.ecma-international.org/publications/standards/Ecma-334.htm}},
como parte fundamental de su plataforma .NET\footnote{\url{http://www.microsoft.com/net/}}.

\begin{itemize}
  \item Los requerimientos de recursos no parecen que sea la mejor opción para una tarea
	de estas características.
  \item Se dispone de SemWeb\footnote{\url{http://razor.occams.info/code/semweb/}}, una
	biblioteca que un inmaduro soporte para RDF y SPARQL. También existen los bindings 
	de Redland\footnote{\url{http://librdf.org/docs/csharp.html}}, aunque ofrencen también
	un pobre rendimiento.
  \item Por ahora no parece existir ninguna biblioteca que ayude en el parseo de los mailboxes 
	de Unix, aunque no parece complicado su desarrollo dada la cantidad de modulos para 
	manejar formatos de correo.
\end{itemize}

Dispone además de varias implementaciones libres, como 
Mono\footnote{\url{http://www.mono-project.com/}} o DotGNU\footnote{\url{http://dotgnu.org/}}.
Pero, al igual que en el caso de Java, hoy por hoy la implementación más completa es la
desarollada por Microsoft. Usar por tanto su framework no sólo complicaria los términos
de distribución de SWAM, sino que encima coartarían su funcionamiento a las plataformas
soportadas actualmente por ese framework (Microsoft Windows).


\paragraph{Perl}: Perl\footnote{\url{http://www.perl.org/}} es un lenguaje de script de 
gran tradición. Soporta paradignas de programación imperativo (estructurados y orientados 
a objetos) y lógico funcionale.

\begin{itemize}
  \item Está especialmente extendido en sistemas Unix y, en menor medida, en sistemas
	operativos Windows. Sus requerimientos sonn realmente bajos y, dada su naturaleza
	de script, está especialmente pensado para invocarse en linea.
  \item Con RDFStore\footnote{http://rdfstore.sourceforge.net/} se dispone de un API bastante
	bueno para manejar RDF desde Perl. También existe una
	implementación\footnote{\url{http://www.w3.org/1999/02/26-modules/}} 
	desarrollada por el W3C para manejar RDF desde Perl. Aunque ni es una implementación
	demasiado completa ni es un proyecto mantenido en la actualidad.
  \item En CPAN\footnote{\url{http://www.cpan.org/}} hay disponibles multitud de bibliotecas
	y módulos útiles para hacer desarrollos en Perl. Entre ellas está
	MessageParser\footnote{\url{http://search.cpan.org/~dcoppit/Mail-Mbox-MessageParser-1.4005/lib/Mail/Mbox/MessageParser.pm}},
	que podría ser una perfecta candidata para resolver en Perl este problema.
\end{itemize}

En su contra juega su sintáxis excesivamente críptica, que hacen muy complicada
la lectura y/o reescritura del código.

\newpage

\paragraph{Conclusión:} Una vez estudiadas y evaluadas cuidadosamente todas estas 
alternativas, se llegó a la conclusión de que Python era el lenguaje que mejor se 
adaptaba a los requisitos del proyecto. Tanto por cumplir los tres requisitos 
no funcionales principales buscados, como por ser un lenguaje moderno y flexible 
que permitirá manejar de una forma muy cómoda todos las estructuras de datos que se
necesitarán.

La documentación es variada, desde la propia página Web oficial del 
lenguaje\footnote{\url{http://www.python.org/}} hasta la gran cantidad de libros
que hay disponibles (\emph{Learning Python}\cite{LearningPython},
\emph{Python Essential Reference}\cite{PythonEssential}, 
\emph{Dive into Python}\cite{DivePython} o \emph{Python Pocket Reference}\cite{PythonPocket},
por ejemplo).
