
\subsection{Visión}

Como ya se comentó en los objetivos recogidos en el capítulo destinado a
la memoria del proyecto, se tiene como objetivo principal la publicación 
de los archivos antiguos de listas de correo en un formato rico 
semánticamente.

De esta manera esta inmensa base de conocimiento podrá ser procesada a 
posteriori por aplicaciones ya existentes o nuevas aplicaciones que 
\emph{exploten} dicho enriquecimiento semántico. Esta ya es un área de 
actuación sólo contemplada parcialmente en el proyecto.

\subsubsection{Situación actual}

Dado que que no existe en la actualidad ninguna solución software que
resuelva el problema, la herramienta cubriría una serie de necesidades
aún sin explotar, que seguramente abran un nuevo camino a nuevos 
desarrollos en este campo.

\subsubsection{Descripción del problema}

Se pretende solucionar un problema principalmente:

\begin{itemize}
  \item \textbf{El problema de} exportar semánticamente los archivos 
	de una lista de correo.
  \item \textbf{Afecta a} los usuarios que habitualmente utilizan los 
	archivos de estas listas de correo como completa fuente de 
	información.
  \item \textbf{Lo que implica} es obtener de mbox todos los mensajes 
	y sus relaciones, para poder publicarlas sin apenas pérdida de
	información.
  \item \textbf{Una solución adecuada haría que} fueran muchos más
	explotables estos datos, permitiendo por ejemplo búsquedas
	mucho más completas y fiables.
\end{itemize}

\subsubsection{Descripción de los usuarios}

El software tendrá dos tipos de usuarios muy claramente identificados:

\begin{enumerate}
  \item \textbf{Usuario administrador:} será la persona encargada de
	instalar el software y parametrizarlo para que realice su
	función según sus necesidades particulares. Evidentemente debe
	ser un usuario avanzado con conocimientos básicos de administración
	de sistemas.
  \item \textbf{Usuarios convencionales:} los usuarios que \emph{consuman},
	bien con alguna aplicación genérica u otra cualquiera hecha a medida,
	el conocimiento generado para cubrir una necesidad concreta. No
	tendrían, en principio, que necesitar tener ningún conocimiento 
	sobre la materia, ni siquiera fuertes conocimientos informáticos.
\end{enumerate}

\subsubsection{Resumen de capacidades}

A continuación se identifican las capacidades del producto en términos de 
beneficios para el usuario y la característica que lo proporciona:

\begin{itemize}
  \item \textbf{Recomponer la lista de correo:} se podrá recomponer, filtrando 
	previamente por múltiples parámetros como fecha o tema, los hilos de una 
	lista de correo sin necesidad de estar suscrito a ella
  \item \textbf{Obtener más información de los suscriptores:} apoyándose por 
	ejemplo en FOAF, se podrán conocer muchos detalles de los suscriptores 
	que la lista de correo no contiene en su formato original.
  \item \textbf{Mejora de la accesibilidad:} intrínsecamente al describir muy 
	ricamente el contenido, le será muy fácil interpretar correctamente 
	la información a los agente de usuario para personas que sufren algún 
	tipo de discapacidad.
\end{itemize}

\subsubsection{Calidad exigida al producto}

Se definen unos rangos de calidad respecto a eficiencia, robustez, 
tolerancia a fallos, facilidad de manejo y características similares 
del sistema software a desarrollar.

\begin{itemize}
  \item \textbf{Disponibilidad:} el software generará una base del 
	conocimiento que debe estar accesible de forma continua.
  \item \textbf{Escalabilidad:} la arquitectura general del sistema 
	será lo suficientemente flexible para soportar un amplio
	rango en los datos que pueda manejar, siendo extremadamente
	recomendable contemplar optimizaciones de generación de los 
	datos de forma incremental.
  \item \textbf{Mantenimiento:} evidentemente tanto la solución software 
	como en los datos generados se ha de tener muy en cuenta su
	mantenibilidad.
\end{itemize}




