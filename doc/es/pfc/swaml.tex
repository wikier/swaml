\documentclass[spanish,a4paper,10pt]{report}
\usepackage[latin1]{inputenc}
\usepackage[T1]{fontenc}
\usepackage[margin=1in]{geometry}
\usepackage{paralist}
\usepackage{graphicx}
\usepackage{url}
\usepackage{times}
\usepackage{babel}
\usepackage{listings}
\usepackage{verbatim}

\pagestyle{headings}


%listings

\definecolor{darkred}{rgb}{0.5, 0, 0}
\definecolor{violet}{rgb}{1, 0, 1}
\definecolor{green}{rgb}{0.3, 0.95, 0.3}
\definecolor{listinggray}{gray}{0.97}

\lstset{
	basewidth=0.50em,
	backgroundcolor=\color{listinggray},
	basicstyle=\footnotesize\ttfamily,
	keywordstyle=\bfseries,
	stringstyle=\itshape,
	commentstyle=\itshape,
	showspaces=false,
	showtabs=false,
	showstringspaces=false,
	frame=trbl,
	extendedchars=true,
	numbers=none,
	aboveskip=0.5cm,
	belowskip=0.5cm,
	xleftmargin=0cm,
	xrightmargin=0cm
}

\lstdefinelanguage{mbox}{%no funciona!
	morekeywords = {From, Message, Date, Organization, To, Subject }
}

\defverbatim[colored]\MBOX{
\lstset{language=mbox}
\begin{lstlisting}
...
From sioc-dev@googlegroups.com Fri Sep 15 13:35:44 2006
Message-ID: <1158352519.450b0e871c79e@courrier.privatedns.com>
Date: Fri, 15 Sep 2006 16:35:19 -0400
From: Frederick Giasson <fred@fgiasson.com>
To: sioc-dev@googlegroups.com
Subject: Implementation of the SIOC v1.08 ontology in Talk Digger
...
From sioc-dev@googlegroups.com Tue Sep 19 07:10:22 2006
From: Kjetil Kjernsmo <kjetilk@opera.com>
Organization: Opera Software ASA
To: sioc-dev@googlegroups.com
Subject: Re: User vs. Person complexity
Date: Tue, 19 Sep 2006 16:09:15 +0200
...
\end{lstlisting}
}


\title{%
	SWAML, publicaci�n de listas de correo en Web Sem�ntica}
\author{Sergio Fern\'andez L\'opez}
\date{Noviembre de 2006}

\begin{document}

\maketitle

\bibliographystyle{plain}

\chapter*{Resumen}

Este documento contiene la documentaci�n del proyecto titulado
\emph{SWAML, publicaci�n de listas de correo en Web Sem�ntica}, 
presentado por el autor para la obtenci�nn del t�tulo de Ingeniero 
T�cnico en Inform�tica por la Escuela Universitaria de Ingenieria 
T�cnica en Inform�tica de Oviedo (Universidad de Oviedo).

El objetivo del proyecto es publicar listas de correo en formatos
con sem�ntica (principalmente RDF), para investigar y completar
determinada informaci�n que no es posible conseguir con los formatos
de publicaci�n actuales.

La p�gina web del proyecto es \url{http://swaml.berlios.de/}

\section*{Palabras clave}

Web Sem�ntica, RDF, OWL, python, listas de correo, mbox.

\chapter*{Agradecimientos}

FIXME

\chapter*{Licencia}

\section*{Documento}

El contenido de este documento se encuentra protegido por la licencia 
\emph{Creative Commons Reconocimiento 2.5} (anexo \ref{sec:license.cc}).

\section*{C�digo fuente}

El c�digo fuente (disponible en el anexo \ref{sec:source}) se encuentra 
licenciado bajo la licencia \emph{GNU General Public License (GPL)}, 
versi�n 2 o superior (anexo \ref{sec:license.gpl}).

\chapter*{Historial de este documento}

\begin{tabular}{|l|l|l|}
 \hline
 \textbf{Fecha} & \textbf{Versi�n} & \textbf{Comentarios} \\\hline
 Jun/2006 & 0.1 & Primer borrador \\\hline
 Ago/2006 & 0.2 & Segundo borrador \\\hline
 Nov/2006 & 0.3 & Tercer borrador con el �ndice definitivo \\\hline
\end{tabular}

\tableofcontents
\newpage
\listoffigures
\newpage


\chapter{Memoria}


\section{Introducci�n}

Los archivos de las listas de correo (es decir, los mensajes antiguos) son
frecuentemente publicados en la web e indexados por los buscadores convencional
es. La base de conocimientos que introducen en la web es enorme.

Sin embargo, una gran cantidad de informaci�n se pierde durante la publicaci�n,
con el resultado de que los archivos publicados son inc�modos de consultar
y poco funcionales.

Este documento describe la aplicaci�n de la web sem�ntica para evitar la
p�rdida de informaci�n y habilitar la construcci�n de nuevas aplicaciones
para explotar m�s convenientemente la informaci�n.


\newpage


\section{Objetivos}

Los objetivos son los recogidos en el documento que contiene la propuesta 
inicial (reproducida en el anexo \ref{sec:propuesta}) redactada por Diego 
Berrueta.

\subsection{Objetivo principal}

Por tanto este proyecto tiene un objetivo principal:

\begin{itemize}
  \item La publicación de los archivos antiguos de listas de correo en un 
	formato rico semánticamente, de manera que pueda ser procesado y
	aprovechado este volumen ingente de conocimiento.
\end{itemize}

\subsection{Objetivos secundarios}

Pero además implícitamente hay otra serie de objetivos secundarios que el
proyecto debe cubrir:

\begin{itemize}
  \item Maximizar la reutilización de la infraestructura disponible previamente
	(herramientas, ontologías, etc), de manera que su uso sirva para mejorarla
	y hacerla crecer.
  \item Desarrollar un prototipo capaz de recomponer la lista de correo 
	\emph{atacando} las colecciones de ficheros RDF previamente exportadas,
	así como hacer sencillas búsquedas en términos que ahora no son
	posibles contemplar.
  \item Abrir la puerta a nuevas aplicaciones construidas sobre esta nueva forma
	de publicar los archivos de las listas de correo.
\end{itemize}



\newpage


\chapter{La Web Sem�ntica}

En 1989 Tim Berners-Lee realiz� para el CERN un modelo de gesti�n de la 
informaci�n basado en un sistema distribuido de hipertexto\footnote{\url{http://www.w3.org/Proposal}}. 
Fue el origen de lenguaje de marcado HTML y la semila de la Web actual
que recientemente ha cumplido 15 a�os\footnote{\url{http://www.sun.com/aboutsun/media/features/www15.html}}.

No tard� mucho en darse cuenta que ese modelo no era suficiente para 
manejar grandes vol�menes de informaci�n, pues resultar�a muy dificil 
encontrarla y usarla eficientemente. As� el propio Tim Berners-Lee 
expondr�a\footnote{\url{http://www.scientificamerican.com/article.cfm?articleID=00048144-10D2-1C70-84A9809EC588EF21&catID=2}}
en 2001 su visi�n\footnote{Cita extraida de las transparencias de Jos� Emilio 
Labra Gayo para el curso de verano sobre Web Sem�ntica de la Universidad de 
Oviedo, \url{http://www.di.uniovi.es/~labra/cursos/ver06/}} de la Web Sem�ntica:

\begin{quote}
	\emph{�... \textbf{disponer datos} en la Web \textbf{definidos y enlazados} 
	de forma que puedan ser \textbf{utilizados por las m�quinas}, no solamente 
	para visualizarnos, sino tambi�n para \textbf{automatizar} tareas, 
	\textbf{integrar} y \textbf{reutilizar} datos entre aplicaciones.�}
\end{quote} 

Y quiz�s se est� dedicando mucho esfuerzo a publicar y procesar de forma aut�noma
esos datos, obviando quiz�s la parte m�s importante: 
\textbf{enlazarlos}\footnote{Traducci�n libre de un extracto de un documento (\url{http://www.w3.org/DesignIssues/LinkedData}) de Tim Berners-Lee}.

\begin{quote}
	\emph{�La Web Sem�ntica no s�lo se trata de publicar datos en la Web. Se 
	trata de enlazarlos para que personas o m�quinas podamos explorar esos 
	datos. Al estar enlazados, podremos encontrar f�cilmente datos relacionados 
	con los datos que disponemos.�}
\end{quote}

\section{Evoluci�n de la Web}

La Web es un recurso muy especial y particular, con unas caracter�sticas muy 
especiales que deben tenerse en cuenta: no centralizada, informaci�n din�mica,
mucha cantidad de informaci�n y est� abierta a todo el mundo.

En los �ltimos a�os la Web ha experimentado una notable evoluci�n, como puede 
verse en la figura \ref{fig:evoWeb}\footnote{Extraida de una presentaci�n 
(\url{http://www.w3c.es/Presentaciones/2005/1018-WebSemanticaREBIUN-MA/}) de 
Mart�n �lvarez Espinar, de la Oficina Espa�ola del W3C}, que le ha llevado a 
convertirse no s�lo en un almacen de contenido est�tico, sino tambi�n en un 
repositorio universal de conocimiento y servicios.

\begin{figure}[ht]
	\centering
	\includegraphics[width=12cm]{images/web-evolution.png}
	\caption{Evoluci�n de la Web}
	\label{fig:evoWeb}
\end{figure}

A pesar de todas estas peque�as revoluciones, todavia seguimos en una web 
\emph{sint�ctica}. Una Web en la que todav�a es muy dif�cil realizar
muchas tareas que con la Web Sem�ntica, y todas sus tecnolog�as, al menos
ser�n un poco m�s f�cil de hacer.

\section{Estructura de la Web Sem�ntica}

La Web Sem�ntica se encuentra estructurada en capas (la llamada \emph{tarta de 
la Web Sem�ntica}), de forma que se pudiera trabajar en cada uno de estos
sustratos de manera independiente a el estado de la implementaci�n de las
capas inferiores y/o superiores.

Algunas parte del dise�o a�n se estan discutiendo en los distintos grupos de
trabajo, aunque en la figura \ref{fig:swStack} encontramos el dise�o que toma
m�s forma despu�s de los �ltimos a�os de trabajo:

\begin{figure}[ht]
	\centering
	\includegraphics[width=10cm]{images/semantic-web-stack.png}
	\caption{Pila de la web sem�ntica}
	\label{fig:swStack}
\end{figure}

Puede verse que el nucleo de la Web Sem�ntica se fundamenta sobre tres 
tecnolog�as fundamentales:

\begin{itemize}
  \item RDF
  \item OWL
  \item SPARQL
\end{itemize}

Con una envoltura de l�gica y reglas, sustentadas sobre una infraestructura 
basada en XML, URI's y Unicode.

Ve�se que las reglas pasan a estar a mismo nivel que las ontolog�as, y no por
encima como se pensaba en los primeros dise�os, por la �ntima relaci�n que ambas
tienen.

\section{Elementos}

\subsection{Elementos b�sicos}

\subsubsection{Unicode}

Unicode\footnote{\url{http://www.unicode.org/}} es una iniciativa de un consorcio de 
empresas dedicadas a la internacionalizaci�n para conseguir una representaci�n 
inform�tica de los caracteres en todos los idiomas de forma que pueda ser representados
y manipulados de forma universal. 

Aunque existen varias codificaci�nes distintas, 
UTF-8\footnote{\url{http://www.ietf.org/rfc/rfc3629.txt}} es la codificaci�n 
unicode m�s usada y extendida, tanto por su sencillez (usa grupos de bytes) 
como por su flexibilidad (los alfabetos de muchos de los lenguajes del mundo 
se pueden representar en UTF-8).

Unicode es el recurso primario de representaci�n de caracteres en la Web Sem�ntica.

\subsubsection{URI}

Acronimo del ingl�s Uniform Resource Identifier, identificador uniforme de recursos.
Mecanismo que la Web Sem�ntica utiliza para identificar recursos. 

Un concepto que mezcla URL\footnote{Uniform Resource Locator, localizador uniforme de recurso} 
y URN\footnote{Uniform Resource Name, nombre uniforme de recurso} para cumplir una doble 
funcionalidad: servir como protocolo de acceso e identificar de manera �nica los recursos 
en la World Wide Web.

\subsection{XML}

XML\footnote{\url{http://www.w3.org/XML/}} (eXtensible Markup Language, Lenguaje 
de marcado extensible) es un formato de marcado estructurado para la representaci�n 
de informaci�n muy usado y extendido hoy en d�a, hasta tal punto de ser considerado 
el lenguaje universal para el intercambio de informaci�n. 

Desarrollado por el W3C a partir de SGML\footnote{\url{http://www.w3.org/MarkUp/SGML/}} 
con el objetivo que fuera f�cilmente procesable por una m�quina y legible por un 
humano. 

Basandose en una definici�n abstracta (XML Schema), permite extender su gram�tica 
de una forma muy f�cil y sencilla de procesar (XSL/XSLT, XPath, etc).

Es usado en m�ltiples tecnol�g�as hoy en d�a, sobre todo en la web (XHTML, XForms, 
SVG, etc), aunque tambi�n para documentaci�n (DocBook), interfaces de usuario (XUL,
Glade, XAML, etc), protocolos (Jabber), etc.

Pero XML, a pesar de ser un lenguaje estructurado que es muy f�cil de procesar, 
a�n carece de la sem�ntica necesaria.

Se utilizan espacios de nombres (namespace), identificados por URI's, para mezclar en 
un documento etiquetas pertenecientes a diferentes vocabularios

\subsection{RDF}

Acr�nimo del ingl�s \emph{Resource Description Framework} (marco de descripci�n 
de recursos), RDF\footnote{\url{http://www.w3.org/RDF/}} es una especificaci�n del 
W3C originalmente dise�ada como modelo de datos, pero su uso se ha extendido como
m�todo general para modelar el conocimiento.

RDF es un modelo de tripletas del tipo \texttt{(sujecto, predicado, objecto)}. El
sujeto es un recurso que se identifica con una URI, y se relaciona mediante un 
predicado binario con el objeto, que puede ser otra URI o un literal.

\begin{figure}[ht]
	\centering
	\includegraphics[width=10cm]{images/arc.png}
	\caption{Arco RDF}
	\label{fig:rdfTriplet}
\end{figure}

Cada tripleta puede verse como un arco, que al juntarse con otros arcos se obtiene
un grafo dirigido que describe los recursos y las relaciones entre todos los 
recursos.

Un ejemplo sencillo: \textit{Sergio Fdez es el creador de http://www.wikier.org/}. 
Usando Dublin Core (secci�n \ref{sec:dc}) podr�a quedar la siguiente tripleta: 
\texttt{(http://www.wikier.org/, dc:creator, "Sergio Fdez")}. Dando lugar a un 
grafo del estilo de:

\begin{figure}[ht]
	\centering
	\includegraphics[width=10cm]{images/arc-example.png}
	\caption{Ejemplo de arco RDF}
	\label{fig:rdfTripletExample}
\end{figure}


RDF se puede serializar con tres sint�xis: en XML, N3 (notaci�n de tripletas)
o Turtle. El mismo grafo se podr�a serializar con las dos sint�xis, conteniendo ambos 
id�ntica informaci�n sem�ntica:

\begin{figure}
\begin{verbatim}
	<rdf:RDF xmlns:rdf="http://www.w3.org/1999/02/22-rdf-syntax-ns#"
	         xmlns:rdfs="http://www.w3.org/2000/01/rdf-schema#"
                 xmlns:dc="http://purl.org/dc/elements/1.1/"
	>
	  <rdf:Description rdf:about="http://www.wikier.org/">
	    <dc:creator>Sergio Fdez</dc:creator>	
	  </rdf:Description>
	</rdf:RDF>
\end{verbatim}
	\caption{Ejemplo de grafo RDF serializado en XML}
	\label{fig:ejemplo.rdfxml}
\end{figure}

\begin{figure}
\begin{verbatim}
	@prefix dc <http://http://purl.org/dc/elements/1.1/>
	<http://www.wikier.org> dc:creator "Sergio Fdez"
\end{verbatim}
	\caption{Ejemplo de grafo RDF serializado en N3}
	\label{fig:ejemplo.rdfn3}
\end{figure}

\begin{figure}
\begin{verbatim}
	@prefix dc <http://http://purl.org/dc/elements/1.1/>
	<http://www.wikier.org> 
		dc:creator "Sergio Fdez"
\end{verbatim}
	\caption{Ejemplo de grafo RDF serializado en Turtle}
	\label{fig:ejemplo.rdfturtle}
\end{figure}

\subsection{RDFS}

RDFS\footnote{\url{http://www.w3.org/TR/rdf-schema/}} (RDF Schema) es una
forma primitiva y limitada de describir ontolog�as en RDF, tambi�n llamado
�\emph{vocabulario RDF}�.

\subsection{OWL}

Del acr�nimo en ingl�s \emph{Ontology Web Languaje}, 
OWL\footnote{\url{http://www.w3.org/TR/owl-features/}} es la recomendaci�n 
oficial de W3C\footnote{\url{http://www.w3.org/}} para publicar ontolog�as en 
la Web. Se trata de un lenguaje de gran expresividad para describir conceptos 
y relaciones entre conceptos, con un compromiso entre expresividad y tratabilidad

Tiene en DAML+OIL\footnote{\url{http://www.daml.org/2001/03/daml+oil-index}}, 
otros dos lenguajes de ontolog�as, sus antecesores inmediatos.

La versi�n actual de OWL (1.0) tiene principalmente tres variantes seg�n su
complejidad:

\begin{itemize}
  \item OWL Full, intimamente ligado a la l�gica de RDF, pero que puede resultar
	no computable.
  \item OWL DL, un subconjunto del anterior basado en la l�gica descriptiva 
	\begin{displaymath}
		{SHOIN} (D)
	\end{displaymath}
  \item OWL Lite, subconjunto de OWL DL que se basa en la l�gica menos descriptiva
	\begin{displaymath}
 		{SHIF} (D)
	\end{displaymath}
\end{itemize}

Una perpectiva sencilla ser�a mostrada por la figura \ref{fig:owlVariants}.

\begin{figure}[ht]
	\centering
	\includegraphics[width=12cm]{images/owl-variants.png}
	\caption{Perspectiva completa de OWL}
	\label{fig:owlVariants}
\end{figure}

Aunque el panorama no es tan sencillo, y para diferenciar cada una de ellas no
se puede hacer fijandose s�lo en la complejidad, sino tambi�n en la parte de la 
l�gica que abarcan. Quedando un panorama a�n m�s confuso, como se puede ver en la
figura \ref{fig:owlVariantsExtended}\footnote{Gr�fico extraido de Ontotext, 
\url{http://www.ontotext.com/inference/rdfs_rules_owl.html#owl_fragments}}.

\begin{figure}[ht]
	\centering
	\includegraphics[width=10cm]{images/owl-dialects.png}
	\caption{Variantes de OWL ampliado}
	\label{fig:owlVariantsExtended}
\end{figure}

Hay que tener en cuenta que existe cierto solapamiento entre la expresividad de
OWL y la de RDFs (RDF-Schema), desvirtuando en cierta manera la visi�n original
por capas.

\subsection{SPARQL}

SPARQL\footnote{\url{http://www.w3.org/TR/rdf-sparql-query/}} es un nuevo lenguaje 
de consulta sobre la base del conocimiento en OWL/RDF. Actualmente se trata del
candidato a recomendaci�n del W3C por parte del 
DAWG\footnote{\url{http://www.w3.org/2001/sw/DataAccess/}} (RDF Data Access Working 
Group).

Posee una sint�xis similar a la otros lenguajes de consulta relacionales, como
podr�a ser SQL.

\begin{figure}[ht]
	\begin{verbatim}
		PREFIX rdf: <http://www.w3.org/1999/02/22-rdf-syntax-ns#>
		PREFIX dc: <http://purl.org/dc/elements/1.1/>
		SELECT DISTINCT ?x, ?name
		WHERE {
		  ?x dc:creator ?name
		}
	\end{verbatim}
	\centering
	\caption{Ejemplo de consulta SPARQL}
	\label{fig:ejemplo.sparql}
\end{figure}

Aplicando esta sencilla consulta de ejemplo a un RDF, como por ejemplo el de 
antes \ref{fig:ejemplo.rdfxml}, se obtendr�a el nombre del creador de cada recurso
definido.

A pesar de ser un lenguaje relativamente reciente, ya se encuentran disponibles
API's de consulta para numerosos lenguajes de programaci�n: 
RDFLib\footnote{\url{http://rdflib.net/sparql/}} para Python, 
Jena\footnote{\url{http://jena.sourceforge.net/ARQ/}} en Java, 
Redland RDF\footnote{\url{http://librdf.org/}} en C con bindings tambi�n para 
otros lenguajes, twinql\footnote{\url{http://www.holygoat.co.uk/projects/twinql/}} 
en Lisp,etc. Adem�s de estar soportado por alguno de los razonadores m�s conocidos, 
como Pellet\footnote{\url{http://www.mindswap.org/2003/pellet/}} o 
KAON2\footnote{\url{http://kaon2.semanticweb.org/}}.

\section{Aplicaciones pr�cticas}

\subsection{Vocabularios RDF}

Existen multitud de vocabularios RDF para fines muy concretos, desde describir
personas, hasta eventos. Estos vocabularios suelen ser f�cilmente extensibles y 
reutilizables entre si.

Existen multitud de ejemplos:

\subsubsection{Dublin Core\label{sec:dc}}

Dublin Core\footnote{\url{http://dublincore.org/}}, tambi�n conocido por sus siglas DC,
es un vocabulario RDF para la descripci�n de m�ltiples propiedades de todo tipo de 
recursos online.

\subsubsection{RSS}

Desarrollado en el seno de Netscape, RSS es el formato de sindicaci�n de noticias
m�s extendido en la actualidad. Con un complicado historial de versiones\ref{fig:rssEvolution} 
incompatibles entre s� (s�lo la versi�n 1.0 de RSS sean
RDF, el resto de versiones 

\begin{figure}[ht]
	\centering
	\includegraphics[width=10cm]{images/rssEvolution.png}
	\caption{Evolucion de RSS}
	\label{fig:rssEvolution}
\end{figure}

\subsubsection{FOAF}

FOAF\footnote{\url{http://www.foaf-project.org/}}, acr�nimo de Friend-of-a-Friend, 
se trata de un vocabulario RDF para describir sem�nticamente informaci�n personal.

\begin{figure} [ht]
\begin{verbatim}
	<rdf:RDF
		xmlns:rdf="http://www.w3.org/1999/02/22-rdf-syntax-ns#"
		xmlns:rdfs="http://www.w3.org/2000/01/rdf-schema#"
		xmlns:foaf="http://xmlns.com/foaf/0.1/"
	>
	  <foaf:Person rdf:nodeID="me">
	    <foaf:name>Sergio Fern�ndez</foaf:name>
	    <foaf:title>Mr</foaf:title>
	    <foaf:firstName>Sergio</foaf:firstName>
	    <foaf:surname>Fern�ndez</foaf:surname>
	    <foaf:gender>Male</foaf:gender>
	    <foaf:mbox>sergio@wikier.org</foaf:mbox>
	  </foaf:Person>
	</rdf:RDF>
\end{verbatim}
	\caption{Ejemplo de FOAF}
	\label{fig:ejemplo.foaf}
\end{figure}

Pudiendo describir relaciones (amigos, compa�eros de trabajo, etc), inter�ses,
proyectos y dem�s recursos de uso personal, de forma que se forme un grafo 
uniendo todos ellos.

\subsubsection{DOAP}

Description-of-a-Project\footnote{\url{http://usefulinc.com/doap}} es una idea 
similar a FOAF pero para describir proyectos de software libre.

\begin{figure}[ht]
\begin{verbatim}

<rdf:RDF xmlns:rdf="http://www.w3.org/1999/02/22-rdf-syntax-ns#" 
         xmlns:rdfs="http://www.w3.org/2000/01/rdf-schema#" 
         xmlns:doap="http://usefulinc.com/ns/doap#" 
         xmlns:foaf="http://xmlns.com/foaf/0.1/" 
         xmlns:admin="http://webns.net/mvcb/" 
         xml:lang="en">

  <doap:Project rdf:about="http://swaml.berlios.de/">

    <doap:name>Semantic Web Archive of Mailing Lists</doap:name>
    <doap:shortname>SWAML</doap:shortname>

    <doap:homepage rdf:resource="http://swaml.berlios.de/"/>
    <doap:created>2005-09-24</doap:created>

    <doap:description xml:lang="en">
      SWAML is a research project around the semantic web tecnologies 
      to publish the mailing lists's archive into an RDF format.
    </doap:description>
    <doap:description xml:lang="es">
      SWAML es un proyecto de investigaci�n alrededor de las tecnolog�as 
      de la web sem�ntica para publicar los archivos de las listas de 
      correo en un formato RDF.
    </doap:description>

    <doap:wiki rdf:resource="http://swaml.berlios.de/wiki"/>
    <doap:bug-database rdf:resource="http://swaml.berlios.de/bugs"/>
    <doap:programming-language>python</doap:programming-language>
    <doap:license rdf:resource="http://usefulinc.com/doap/licenses/gpl"/>
    <doap:download-page rdf:resource="http://swaml.berlios.de/#files"/>
    <doap:download-mirror rdf:resource="http://swaml.berlios.de/files"/>

  </doap:Project>

</rdf:RDF>
\end{verbatim}
	\caption{Fichero DOAP de SWAML}
	\label{fig:ejemplo.doap}
\end{figure}

\subsubsection{EARL}

FIXME

\subsection{Otras aplicaciones de RDF}

\subsubsection{Mozilla}

FIXME

\section{Libros de inter�s}

Aunque no existen buenas publicaciones en castellano, hay dos libros en ingl�s que
es imprescindible ojear: \emph{A Semantic Web primer}\cite{SemanticWebPrimer} y
\emph{Practical RDF}\cite{OreillyPracticalRDF}.

\section{Futuro}

Hacia donde ir� la Web Sem�ntica es algo que s�lo se sabr� con el tiempo, y quiz�s
en un plazo no m�s all� de 5 o 10 a�os. 

Hist�ricamente todos los campos relacionados con la IA (inteligencia artificial) 
han sido �reas de investigaci�n vendidas en exceso como la panacea de la soluci�n 
de todos los problemas de la humanidad; generando unas espectativas falsas que han 
hecho mucho da�o a la comunidad cient�fica.

Lo que si sabemos es que la Web Sem�ntica ser� una Web mucho m�s \emph{�til}.






\chapter{Metodología}


\section*{RUP: Rational Unified Process}

La metodolog�a utilizada para la elaboraci�n de este proyecto ha sido 
\emph{RUP}\footnote{\url{http://www.rational.com/products/rup/}}
(Rational Unified Process, Proceso Racional Unificado) de IBM.

La raz�n principal fue la genericidad que brinda para el proceso de desarrollo
software, adaptandose perfectamente a desarrollos basados en programaci�n 
orientada a objetos.

Ayudo a implementar determinadas buenas pr�cticas en Ingenier�a del Software:

\begin{itemize}
  \item Desarrollo iterativo
  \item Administraci�n de requisitos
  \item Uso de arquitectura basada en componentes
  \item Control de cambios
  \item Verificaci�n de la calidad del software
\end{itemize}

Por tanto todo el proceso de desarrollo se dividi� en ciclos, teniendo un 
producto final al final de cada ciclo, dividiendo cada ciclo en fases que 
finalizan con un hito importante.

\begin{figure}[ht]
	\centering
	\includegraphics[width=12cm]{images/rup.png}
	\caption{Vista general de RUP}
	\label{fig:RUP}
\end{figure}

Los fases seguidas fueron:

\begin{enumerate}
  \item Inicio: se hizo un plan de fases, identificando los principales 
	casos de uso y riesgos.
  \item Elaboraci�n: se hizo un plan de proyecto, completandose los casos 
	de uso para eliminar los riesgos.
  \item Construcci�n: se concentro en la elaboraci�n de un producto totalmente 
	operativo y eficiente, asicomo en un peque�o manual de usuario.
  \item Transici�n: se implemento el producto en el cliente y se publicaron
	varias versiones funcionales. Como consecuencia de esta publicaci�n
	surgieron nuevos requisitos a ser analizados.
\end{enumerate}

Para cada fase RUP define nueve actividades a realizar:

\begin{enumerate}
 \item Modelado del negocio
 \item An�lisis de requisitos
 \item An�lisis y dise�o
 \item Implementaci�n
 \item Test
 \item Distribuci�n
 \item Gesti�n de configuraci�n y cambios
 \item Gesti�n del proyecto
 \item Gesti�n del entorno
\end{enumerate}

Adem�s de un flujo de trabajo entre ellas:

\begin{figure}[ht]
	\centering
	\includegraphics[width=9cm]{images/workflow-rup.png}
	\caption{Flujos de trabajo de RUP}
	\label{fig:RUP}
\end{figure}

En RUP de utiliza UML\footnote{\url{http://www.omg.org/uml/}} como herramienta principal
para la documentaci�n de toda la arquitectura del sistema. La bibliografia es amplia,
disponiendo de veteranos t�tulos como 
\emph{The Unified Modeling Language Reference Manual}\cite{UMLReference} o 
\emph{UML Distilled}\cite{UMLDistilled} como guias de referencia, siendo imprescidible
tener siempre a mano el \emph{UML 2.0 Pocket Reference}\cite{UMLPocket} para resolver
r�pidamente esas consultas puntuales de la especificaci�n.

RUP puede englobarse dentro de lo que algunos llaman \emph{procesos pesados},
estando quiz�s muy orientado para proyecto de algo m�s de envergadura que SWAML.
Es por ello que debido a las caracteristicas del proyecto, tama�o y personas 
involucradas, se ha adaptado RUP utilizando s�lo los documentos y procesos de 
dise�os necesarios.

Este m�todo es de sobra conocido y esta perfectamente documentado (en libros como
\emph{The Rational Unified Process}\cite{RUPIntro} y
\emph{The Rational Unified Process Made Easy}\cite{RUPEasy}) como para
extenderse m�s reescribiendo dicha documentaci�n.


\newpage


\section{Concepción}

El objetivo de este documento es recoger todos los aspectos relacionados
con la concepción del proyecto 
\emph{SWAML, publicación de listas de correo en web semántica}, realizando
para ellos una captación de requisitos para su posterior análisis.

\subsubsection{Documentos recogidos}

\begin{itemize}
 \item Visión
 \item Especificación de requisitos
 \item Especificación de casos de uso
 \item Especificaciones secundarias
 \item Plan del proyecto
\end{itemize}

\newpage


\subsection{Visión}

Como ya se comentó en los objetivos recogidos en el capítulo destinado a
la memoria del proyecto, se tiene como objetivo principal la publicación 
de los archivos antigüos de listas de correo en un formato rico 
semánticamente.

De esta manera esta inmensa base de conocimiento podrá ser procesada a 
posteriori por aplicaciones ya existente o nuevas aplicaciones que 
\emph{exploten} dicho entiquecimiento semántico. Esta ya es un área de 
actuación sólo contemplada parcialmente en el proyecto.

\subsubsection{Situación actual}

Dado que que no existe en la actualidad ninguna solución software que
resuelva el problema, la herramienta cubriría una serie de necesidades
aún sin explotar, que seguramente habrán un nuevo camino a nuevos 
desarrollos en este campo.

\subsubsection{Descripción del problema}

Se pretende solucionar un problema principalmente:

\begin{itemize}
  \item \textbf{El problema de} exportar semánticamente los archivos 
	de una lista de correo.
  \item \textbf{Afecta a} los usuarios que habitualmente utilizan los 
	archivos de estas listas de correo como completa fuente de 
	información.
  \item \textbf{Lo que implica} es obtener de mbox todos los mensajes 
	y sus relaciones, para poder publicarlas sin apenas pérdida de
	información.
  \item \textbf{Una solución adecuada haría que} fueran muchos más
	explotables estos datos, permitiendo por ejemplo busquedas
	mucho más completas y fiables.
\end{itemize}

\subsubsection{Descripción de los usuarios}

El software tendrá dos tipos de usarios muy claramente identificados:

\begin{enumerate}
  \item \textbf{Usuario administrador:} será la persona encargada de
	instalar el software y parametrizarlo para que realice su
	función según sus necesidades particulares. Evidentemente debe
	ser un usuario avanzado con conocimientos básicos de administración
	de sistemas.
  \item \textbf{Usuarios convencionales:} los usuarios que \emph{consuman},
	bien con alguna aplicación genérica u otroa cualquiera hecha a medida,
	el conocimiento generado para cubrir una necesidad concreta. No
	tendrían, en principio, que necesitar tener ningún conocmiento 
	sobre la materia, ni siquiera fuertes conocmientos informáticos.
\end{enumerate}

\subsubsection{Resumen de capacidades}

A continuación se identifican las capacidades del producto en términos de 
beneficios para el usuario y la caracteristica que lo proporciona:

\begin{itemize}
  \item \textbf{Recomponer la lista de correo:} se podrá recomponer, filtrando 
	previamente por múltiples parámetros como fecha o tema, los hilos de una 
	lista de correo sin necesidar de estar suscrito a ella
  \item \textbf{Obtener más información de los suscriptores:} apoyandose por 
	ejemplo en FOAF, se podrán conocer muchos detalles de los suscriptores 
	que la lista de correo no contiene en su formato original.
  \item \textbf{Mejora de la accesibilidad:} intrinsicamente al describir muy 
	ricamente el contenido, le será muy fácil interpretar correctamente 
	la información a los agente de usuario para personas que sufren algún 
	tipo de discapacidad.
  \item FIXME
\end{itemize}

\subsubsection{Calidad exigida al producto}

Se definen unos rangos de calidad respecto a eficiencia, robustez, 
tolerancia a fallos, facilidad de manejo y características similares 
del sistema software a desarrollar.

\begin{itemize}
  \item \textbf{Disponibilidad:} el software generará una base del 
	conocimiento que debe estar accesible de forma continua.
  \item \textbf{Escalibilidad:} la arquitectura general del sistema 
	será lo suficientemente flexible para soportar un amplio
	rango en los datos que pueda manejar, siendo extremadamente
	recomendable contemplar optimizaciones de generación de los 
	datos de forma incremental.
  \item \textbf{Mantenimiento:} evidentemente tanto la solución software 
	como en los datos generados se ha de tener muy en cuenta su
	mantenibilidad.
\end{itemize}






\newpage


\subsection{Especificación de requisitos}

Debido a la naturaleza del problema a resolver la \textbf{introspección}
ha sido la técnica usada para realizar la captura de requisitos software.

Esta técnica recomienda que sea el propio ingeniero de requisitos quien 
se ponga en el lugar del cliente y trate de imaginar como desearía él el 
sistema. Y en base a estas suposiciones comenzar a recomendar al cliente 
sobre la funcionalidad que debería presentar el sistema. El problema radica 
en  que un ingeniero no es un tipo normal de cliente, posee un conocimiento 
técnico mas elevado por lo que se podrían recomendar cosas que el cliente 
no necesite.

Pero además circunstancialmente también se hizo uso de la técnica conocida
como las \textbf{entrevistas}, principalmente discusiones. Como adaptación
a las circunstancias concretas fueron ambos co-directores del proyecto 
quienes ejercieron la función del cliente.

\subsubsection{Datos de entrada}

La fuente de información será una lista de correo en formato
mbox\footnote{\url{http://rfc.net/rfc4155.html}}, un formato estandarizado
que utilizan la mayoria de los sistemas de gestión de listas de correo
(Mailman\footnote{\url{http://www.gnu.org/software/mailman/}}, 
Majordomo\footnote{\url{http://www.greatcircle.com/majordomo/}},
LISTSERV, Listproc y SmartList entre otros).

El formato mbox no es más que un fichero de texto plano en el que se van
almacenando consecutivamente los correos que van llegando a la lista. Se 
almacenan tal cual son enviados a la lista, con su cabeceras originales 
completas y en la codificación del cliente de correo del usuario.

Por tanto es fácil adivinar dos problemas evidentes de este formato:

\paragraph{Identificacores}

FIXME

Cada correo dispone de un identificador (cabecera \texttt{Message-Id}). Cuando
alguien responde un mensaje, el cliente de correo colocará en el nuevo una 
cabecera (\texttt{In-Reply-To}) con este ID para referirse explicitamente al 
mensaje que se está respondiendo.

Este el mecanismo especificado en el RFC\footnote{\url{http://rfc.net/rfc2822.html}} 
para las gestión de hilos de conversación por medio de correo electrónico. Y es
mecanismo que utilizado para representar las conversaciones en forma de árbol, 
tanto en clientes de correo (Evolution, Thunderbird, Outlook, etc) como en sistemas 
convecionales de publicación de listas de correo.

Dicho identificador tiene una forma similar a \texttt{<3C94C55A3B6A@smtp.isp.com>}.

Pero ese ID no es único, sino que es asignado por el propio servidor
SMTP\footnote{\url{http://es.wikipedia.org/wiki/SMTP}} (Simple Mail Transfer Protocol) 
de forma arbitraria a la hora de enviar el correo.

Por tanto no debería poder usarse, al menos garantizando un rigurosidad extrema
a la hora de identificar cada uno de los mensajes y sus respuestas.

Podría usarse algoritmos eurísticos con el asunto del mensaje, aunque tampoco nos
garantizarían una fiabilidad absoluta al poder cambiarse el asunto en cualquier
mensaje del hilo.

Pero si existe una aproximación al problema que consigue una efectividad bastante
alta según se ha podido comprobar. Consiste en asumir que cuando hay una respuesta
a un mensaje, existe una alta probabilidad que se esté respondiendo a último de los 
mensajes enviados con ID repetido.

\paragraph{Codificación}

FIXME

(cada uno de su padre y de su madre, y almacenados en la codificación del
servidor)

\subsubsection{Datos de salida}

FIXME(HTML, XML, RDF, KML, etc...)

\subsubsection{Lenguaje de programación}

El problema planteado requiere de una lenguaje de programación que disponga
de determinadas características:

\begin{itemize}
  \item Fácil despliegue: hay que procurar que SWAML se pueda desplegar en
	todo tipo de máquinas, sin excesivos requisitos ni hardware ni software.
	Es importante que SWAML pueda ser invocado por los distintos programadores
	de tarea de que disponen los sistemas operativos (cron y similares).
  \item API para RDF: que disponga de una madura bliblioteca, a poder ser nativa, 
	para manejar RDF (creación de grafos, parseo desde disco/URI, serializado
	a disco y/o bases de datos, consultas SPARQL, etc).
  \item Biblioteca para ficheros mbox: sería interesante disponer de una biblioteca 
	que abstraiga lo mayor posible al proyecto del manejo de ficheros
	mbox\footnote{\url{http://rfc.net/rfc4155.html}} y mensajes de correo 
	electrónico\footnote{\url{http://rfc.net/rfc2822.html}}.
\end{itemize}

Por tanto el cumplimiento de estas tres necesidades principales era lo primero
a valorar entre todos lenguajes de programación candidatos a convertirse en el
lenguaje utilizado para implementar SWAML. 

Pero tambien se iba a tener en cuenta otras cualidades más generales al problema,
como por ejemplo:

\begin{itemize}
  \item Aspectos concretos de la OOP (object-oriented programming, programación 
	orientada a objetos) que cubra.
  \item Sencillez de desarrollo y posterior estudio del código.
  \item Portabilidad de la solución generada.
  \item Posibilidad de usarse compiladores/intérpretes libres.
  \item FIXME
\end{itemize}

Despues de revisar los lenguajes dsiponibles, fueron varios los candidatos para
someterlos a un estudio más profundo:

\paragraph{Java:}Java\footnote{\url{http://java.sun.com/}}
es un lenguaje de programación, desarrollado por Sun Microsystems, orientado a 
objetos muy popular desde hace varios años. Java no se compila a código nativo, 
sino que una JVM (Java Virtual Machine, máquina virtual de Java) ejecuta el 
bytecode previamente compilado.

En la actualidad se disponen de multitud de implementaciones de la máquina virtual
de Java, desde las propietarias (Sun, IBM, HP, etc) hasta las libres (Harmony, GIJ, 
Kafee, IKVM.NET, etc).

Sobre el problema que nos atañe:

\begin{itemize}
  \item Actualmente las JVM existentes cubren un amplio abanico de arquitecturas y 
	sistemas operativos. Aunque Java esté más pensado para su uso en otro tipo
	de entornos (J2EE por ejemplo), puede invocarse perfectamente en modo en
	linea.
  \item Dispone de forma nativa (desarrollado tambien en Java) de la biblioteca para
	manejar RDF más madura actualmente: Jena\footnote{\url{http://jena.sourceforge.net/}}.
	El famework Jena incluye paquetes para múltiples propósitos dentro de la web
	semántica: API para RDF y OWL, persistencia, serializado y soporte para consultas
	SPARQL.
  \item FIXME
\end{itemize}

Quizás la única (y mayor) pega se encuentre en los términos de licencia de las 
mejores implementaciones de la máquina virtual de Java, lo que complicaria de una
manera importante un futura distribución de SWAML de manera totalmente libre,
pues tendría como dependencias paquetes no libres.


\paragraph{Python:}Python\footnote{\url{http://www.python.org/}} es un lenguaje de 
script extremadamente eficiente. Su uso está muy extendido en todos los sistemas 
Unix actuales (GNU/Linux, familia BSD, Solaris, etc), aunque también está disponible\footnote{\url{http://www.python.org/download/}} para la mayoria de sistemas 
operativos actuales (Windows, MacOS y demás).

Se trata de un lenguaje de script mucho más moderno que otros lenguajes hermanos 
tipo bash o perl. Python va más alla, disponiendo en un lenguaje de script 
interpretado y con tipado dinámico de toda la potencia de los lenguajes orientados 
a objetos más modernos.

Respecto a los tres requisitos que nos interesan:

\begin{itemize}
  \item Al tratarse de un lenguaje de script basta dsiponer de un interprete 
	instalado en el sistema para poderlo ejecutar. Además esta caracteristica
	simplifica enormemente la forma de invocarlo desde un programador de
	tareas.
  \item Existen varias posibilidades para manejar RDF desde Python. Algunas son
	bibliotecas nativas desarrolladas también en Python, y otras están
	disponibles en forma de bindings a bibliotecas desarrolladas en otro 
	lenguaje.
	De todas las posibilidades, quizás RDFLib\footnote{\url{http://rdflib.net/}}
	sea la que se encuentra en un estado de desarrollo más avanzado y maduro 
	(persistencia, serialización, consultas SPARQL, etc).
	Además ofrece la posibilidad de \emph{colocar encima} otras bibliotecas,
	como por ejemplo Sparta\footnote{\url{http://www.mnot.net/sw/sparta/}},
	para utilizar determinados conceptos que no contempla RDFLib.
  \item Python dispone una extensa y completa biblioteca estandar, además de contar
	con multitud de bibliotecas para los más variopintos propósitos. Con modulos 
	como mailbox\footnote{\url{http://docs.python.org/lib/module-mailbox.html}}
	e email\footnote{\url{http://docs.python.org/lib/module-email.html}}, el
	problema de acceso primario a los datos (mailbox unix) que SWAML deberá
	consumir se verá resuelto de manera muy eficiente a un nivel de abstracción
	bastante alto.
\end{itemize}

Además es un lenguaje totalmente libre, desde su especificación hasta varias
de sus implementaciones, incluido el interprete oficial.

Mailman\footnote{\url{http://www.gnu.org/software/mailman/}}, el sistema de gestión
de listas de correo más popular hoy en día, también esta escrito en Python, lo que 
facilitaria en gran medida una supuesta integración de SWAML en Mailman.


\paragraph{C\#:}C\#\footnote{\url{http://msdn2.microsoft.com/en-us/vcsharp/aa336809.aspx}} 
es un lenguaje de programación desarrollado por Microsoft, y posteriormente estandarizado por el ECMA\footnote{\url{http://www.ecma-international.org/publications/standards/Ecma-334.htm}},
como parte fundamental de su plataforma .NET\footnote{\url{http://www.microsoft.com/net/}}.

\begin{itemize}
  \item Los requerimientos de recursos no parecen que sea la mejor opción para una tarea
	de estas características.
  \item Se dispone de SemWeb\footnote{\url{http://razor.occams.info/code/semweb/}}, una
	biblioteca que un inmaduro soporte para RDF y SPARQL. También existen los bindings 
	de Redland\footnote{\url{http://librdf.org/docs/csharp.html}}, aunque ofrencen también
	un pobre rendimiento.
  \item Por ahora no parece existir ninguna biblioteca que ayude en el parseo de los mailboxes 
	de Unix, aunque no parece complicado su desarrollo dada la cantidad de modulos para 
	manejar formatos de correo.
\end{itemize}

Dispone además de varias implementaciones libres, como 
Mono\footnote{\url{http://www.mono-project.com/}} o DotGNU\footnote{\url{http://dotgnu.org/}}.
Pero, al igual que en el caso de Java, hoy por hoy la implementación más completa es la
desarollada por Microsoft. Usar por tanto su framework no sólo complicaria los términos
de distribución de SWAM, sino que encima coartarían su funcionamiento a las plataformas
soportadas actualmente por ese framework (Microsoft Windows).


\paragraph{Perl:}Perl\footnote{\url{http://www.perl.org/}} es un lenguaje de script de 
gran tradición. Soporta paradignas de programación imperativo (estructurados y orientados 
a objetos) y lógico funcionale.

\begin{itemize}
  \item Está especialmente extendido en sistemas Unix y, en menor medida, en sistemas
	operativos Windows. Sus requerimientos sonn realmente bajos y, dada su naturaleza
	de script, está especialmente pensado para invocarse en linea.
  \item Con RDFStore\footnote{\url{http://rdfstore.sourceforge.net/}} se dispone de un 
	API bastante bueno para manejar RDF desde Perl. También existe una
	implementación\footnote{\url{http://www.w3.org/1999/02/26-modules/}} 
	desarrollada por el W3C para manejar RDF desde Perl. Aunque ni es una implementación
	demasiado completa ni es un proyecto mantenido en la actualidad.
  \item En CPAN\footnote{\url{http://www.cpan.org/}} hay disponibles multitud de bibliotecas
	y módulos útiles para hacer desarrollos en Perl. Entre ellas está
	MessageParser\footnote{\url{http://search.cpan.org/~dcoppit/Mail-Mbox-MessageParser-1.4005/lib/Mail/Mbox/MessageParser.pm}},
	que podría ser una perfecta candidata para resolver en Perl este problema.
\end{itemize}

En su contra juega su sintáxis excesivamente críptica, que hacen muy complicada
la lectura y/o reescritura del código.

\newpage

\paragraph{Conclusión:}Una vez estudiadas y evaluadas cuidadosamente todas estas 
alternativas, se llegó a la conclusión de que Python era el lenguaje que mejor se 
adaptaba a los requisitos del proyecto. Tanto por cumplir los tres requisitos 
no funcionales principales buscados, como por ser un lenguaje moderno y flexible 
que permitirá manejar de una forma muy cómoda todos las estructuras de datos que se
necesitarán.

La documentación es variada, desde la propia página Web oficial del 
lenguaje\footnote{\url{http://www.python.org/}} hasta la gran cantidad de libros
que hay disponibles (\emph{Learning Python}\cite{LearningPython},
\emph{Python Essential Reference}\cite{PythonEssential}, 
\emph{Dive into Python}\cite{DivePython} o \emph{Python Pocket Reference}\cite{PythonPocket},
por ejemplo).


\newpage


\subsection{Especificación de casos de uso}

FIXME


\newpage


\subsection{Especificaciones secundarias}

\subsubsection{Requisitos del sistema} 

El software no deberá necesitar de unos requerimientos hardware elevados, 
siendo capaz de ejecutarse en un procesador de como mínimo 300MHz, con 
un mínimo de 32Mb de memoria RAM. 

Los requisitos concretos (procesador de 32 o 64 bits, sistema operativo, 
etc) vendrán determinados por las inherentes restricciones del entorno de 
ejecución escogido para el proyecto.

\subsubsection{Requisitos de documentación} 

La documentación aportada deberá contener manuales para un completo uso 
del software entregado.

Al menos los siguientes tres documentos:

\begin{itemize}
  \item \textbf{Manual técnico} en el que se recoja toda la información que 
	fuera necesaria si un futuro se desea extender parte o la totalidad 
	del software por parte de personas totalmente ajenas al equipo de 
	desarrollo original.
  \item \textbf{Manual de despliegue} describiendo detalladamente todos los 
	requisitos previos y los pasos concretos que se deben seguir para la 
	correcta instalación del software.
  \item \textbf{Manual de usuario} que recoja ayuda detallada en un lenguaje
	no técnico para la correcta utilización del software.
\end{itemize}

Todos los documentos deberán, además de ser entregados impresos en papel a la
hora de entrega del resto de componentes del proyecto, estar disponibles en
formato imprimible (como por ejemplo PDF) en el sitio web público del proyecto.


\newpage


\subsection{Plan del proyecto}

\subsubsection{Gráfico de Gantt}

FIXME(planner)

\subsubsection{Personas}

FIXME

\subsubsection{Materiales}

FIXME

\subsubsection{Presupuesto}

FIXME




\newpage


\section{Análisis y diseño (SAD)}

\subsection{Vista de casos de uso}

FIXME

\subsection{Vista lógica}

\subsubsection{Diagrama de clase de SWAML}

\begin{figure}[p]
	\centering
 	\includegraphics[width=14cm]{images/uml/clases/swaml.png}
	\caption{Diagrama de clase de SWAML}
	\label{fig:uml:swaml}
\end{figure}

\subsubsection{Diagrama de clase de Buxon}

\begin{figure}[p]
	\centering
 	\includegraphics[width=15cm]{images/uml/clases/buxon.png}
	\caption{Diagrama de clase de Buxon}
	\label{fig:uml:buxon}
\end{figure}

\subsection{Vista de procesos}

FIXME

\subsection{Vista de implementación}

(describir los paquetes) FIXME

\subsection{Vista de distribución}

FIXME




\newpage


\section{Construcci�n}

FIXME

\subsection{Plan de pruebas}

FIXME

\subsection{Resultado de las pruebas}

FIXME



\chapter{Manuales}

Bajo este epígrafe se incluyen tres manuales de diferente ámbito:

\begin{itemize}
  \item Manual técnico
  \item Manual de despliegue
  \item Manual de usuario
\end{itemize}

\input{manuales/técnico.tex}


\section{Manual de despliegue}

\subsection*{Requisitos técnicos}

El proyecto exige ciertos requisitos técnicos para su correcto
funcionamiento.

\subsubsection*{Intérprete de Python}

Evidentemente será necesario disponer en nuestro sistema de algún intérprete
de Python 2.4 (Python, IronPython u otros...). Se puede 
obtener\footnote{\url{http://www.python.org/download/}} de la propia página 
oficial del lenguaje, aunque se encuentra empaquetada para múltiples sistemas 
operativos. En Debian GNU/Linux\footnote{\url{http://www.debian.org/}}, por 
ejemplo, bastaría con hacer:

\begin{center}
	\texttt{apt-get install python2.4}
\end{center}

\subsubsection*{Bibliotecas necesarias}

Como se puede ver en la sección~\ref{sec:conclu:bib}, han sido varias las
bibliotecas utilizadas en el proyecto que debemos tener instaladas:

\begin{itemize}
  \item RDFLib\footnote{\url{http://rdflib.net/}} = 2.3.1
  \item PyXML\footnote{\url{http://pyxml.sourceforge.net/}}
  \item GTK+\footnote{\url{http://www.gtk.org/}} >= 2.6.0
  \item PyGTK\footnote{\url{http://www.pygtk.org/}} >= 2.6.0
  \item gazpacho\footnote{\url{http://gazpacho.sicem.biz/}} >= 0.6.6
\end{itemize}

La forma de instalarlas ya dependerá del sistema operativo que utilice.

\subsection*{Descomprimir SWAML}

SWAML se distribuye\footnote{\url{http://swaml.berlios.de/files}} comprimido
en ficheros \texttt{.tar.gz}  que podrá descomprimir con casi cualquier 
software de descompresión de ficheros (gzip, tar, WinZip, WinRAR, etc). En 
los propios tarballs\footnote{Nombre por el que conoce los ficheros \texttt{.tar.gz}}
se distribuye esta misma documentación (en un fichero llamado \texttt{INSTALL})
de forma un poco más abreviada.

\subsection*{Instalar SWAML}

SWAML puede ser usado perfectamente si necesidad de instalarse. Aún así la instalación 
de SWAML se realiza con una simple regla del Makefile:

\begin{center}
	\texttt{make install}
\end{center}

\subsection*{Desinstalar SWAML}

También para desinstalarlo basta invocar una simple regla de Makefile:

\begin{center}
	\texttt{make uninstall}
\end{center}



\section{Manual de usuario}

La aplicación entregada se compone en realidad de cinco partes:

\begin{itemize}
 \item SWAML
 \item configWizard
 \item FOAF Enricher
 \item KML Exporter
 \item Buxon
\end{itemize}

Cada uno de estas partes toman la forma de un script Python que puede ser 
invocado mediante su intérprete. La ayuda de cada uno de ello se encuentra
disponible llamandolo con la opción \texttt{--help}, aunque se pasará a 
explicar con más detalle el uso de cada una de estas cinco aplicaciones.

\subsection*{SWAML}

Es la aplicación principal, la que desarrolla el propósito principal del
proyecto. Su funcionalidad se provee por medio del script \texttt{swaml.py}.
Su uso es bien sencillo: como se puede ver en la captura de la 
figura~\ref{fig:swaml} se le invoca acompañado de un único parámetro obligatorio 
que indica la ruta donde esta la configuración que se le quiere pasar a 
SWAML. Inmediatamente se desemboca todo el proceso sin interacción
alguna con el usuario más que las estadísticas informativas que se 
imprimen al final de cada fase importante del proceso. Si no se
imprime ningún error el proceso habrá concluido satisfactoriamente.

\begin{figure}[H]
	\centering
	\includegraphics[width=10cm]{images/screenshots/swaml.png}
	\caption{SWAML}
	\label{fig:swaml}
\end{figure}

\subsection*{configWizard}

Por medio del script \texttt{congWizard.py} se provee un asistente para
ayudar al usario a crear ficheros de configuración según el formato
que debe recibir SWAML. Tal y como se puede ver en la captura de pantalla
de la figura~\ref{fig:configWizard} el script recibe la ruta destino del 
fichero donde se quiera guardar la configuración. El proceso es sencillo: 
el asistente va pidiendo una serie de parámetros al usuario, ofreciendole
un valor por defecto, hasta que haya recopilado toda la información necesaria,
volcándolos inmediatamente a disco con el formato adecuado en la ruta 
indicada por el usuario.

\begin{figure}[H]
	\centering
	\includegraphics[width=14cm]{images/screenshots/configWizard.png}
	\caption{configWizard}
	\label{fig:configWizard}
\end{figure}

Un fichero de ejemplo (que se acompaña con la aplicación) podría ser el
siguiente:

\begin{figure}[H]
\begin{lstlisting}
[SWAML]
title = Example mail list
description = Example description
host = http://example.com/
dir = /var/www/lists/archives/example/
url = http://example.com/lists/archives/example/
mbox = /var/lib/mailman/archives/public/example.mbox
format = YYYY-MMM/postID.rdf
to = example@lists.example.com
kml = yes
foaf = yes
\end{lstlisting}
\caption{Ejemplo de fichero de configuración}
\label{fig:ejemplo-config}
\end{figure}

\subsection*{FOAF Enricher}

Aunque esta funcionalidad se provee en el core de la aplicación principal,
en determinados casos puede ser necesario su uso de manera independiente.
Así el script \texttt{foaf.py} recibe la ruta de un fichero RDF con los
suscriptores de una lista de correo, busca el fichero FOAF de cada uno y
lo enriquece con determinadas propiedades (el propio URI del FOAF, fotografía,
coordenadas geográficas, etc).

\begin{figure}[H]
	\centering
	\includegraphics[width=12cm]{images/screenshots/foaf-enricher.png}
	\caption{FOAF Enricher}
	\label{fig:foaf-enricher}
\end{figure}

\subsection*{KML Exporter}

Al igual que en el caso anterior, la funcionalidad dada por este componente
de manera independiente forma también parte de la aplicación principal. En ese
caso el script \texttt{kml.py} toma como primer parámetro la ruta de un fichero
de suscriptores enriquecido con información geográfica y genera otro fichero
en formato KML posicionando geográficamente los suscriptores.

\begin{figure}[H]
	\centering
	\includegraphics[width=12cm]{images/screenshots/kml-exporter.png}
	\caption{KML Exporter}
	\label{fig:kml-exporter}
\end{figure}

Una manera inmediata para aprovechar esta exportación es utilizar Google Maps para visualizar esos
puntos\footnote{\url{http://maps.google.com/maps?q=http://swaml.berlios.de/demo/subscribers.kml}},
como se puede ver en la figura~\ref{fig:googlemaps}.

\begin{figure}[H]
	\centering
	\includegraphics[width=10cm]{images/screenshots/googlemaps.png}
	\caption{Google Maps}
	\label{fig:googlemaps}
\end{figure}

\subsection*{Buxon}

Buxon es un visor de foros exportados con el vocabulario SIOC. Básicamente 
recibe la URI de un \texttt{sioc:Forum} y recompone en forma de árbol los
hilos de conversación definidos. En la figura~\ref{fig:buxon} se puede ver
una captura de la aplicación en acción, que tiene un buscador parecido con 
los navegadores Web convencionales.

\begin{figure}[H]
	\centering
	\includegraphics[width=16cm]{images/screenshots/buxon.png}
	\caption{Buxon}
	\label{fig:buxon}
\end{figure}

Su uso es bastante sencillo: sólo necesita recibir, bien como parámetro en linea
o en la barra de direcciones habilitada, la URI de un \texttt{sioc:Forum}.
Haciendo click en el boton \textit{"ir a"} la aplicación analizará esa URI y reconpondrá
todos los mensajes que contenga, listándolos en forma de conversación como si se estuviera
viendo con un cliente de correo tradicional. Además el usuario podrá hacer búsquedas
sencillas por contenido y por rango de fechas.





\chapter{Conclusiones}

FIXME

\section{Conclusiones personales}

Este proyecto me ha supuesto muchas cosas a nivel personal y profesional. He
aprendido mucho sobre esta incipiente área que es la Web Semántica; tengo la
suerte de contar con dos directores de proyecto (José Emilio Labra y Diego
Berrueta) que grandes expertos en la materia y me han ayudado en todos estos
meses.

Además me ha brindado la oportunidad de un nuevo lenguaje de programación al
que hacia tiempo le tenía ganas: Python.

Desde el punto de vista ético para mi era muy importante saber que las horas
invertidas en este proyecto no se \emph{morirían} en las polvorientas estanterias
de la biblioteca. Por eso desde eso desde el principio y en todo momento el
proceso de desarrollo de todos los componentes de este proyecto ha estado
disponible en el subversion de Berlios\footnote{\url{http://swaml.berlios.de/wsvn}}.

En estos meses he liberado media docena de versiones de SWAML. Quizás el recorrido
termine aquí, o quizás no. No sé si yo continuaré desarrollando SWAML o si a alguien
le parecerá interesante para continuar el trabajo que yo he comenzado; pero el caso
es que el software liberado ahí seguirá para cualquiera que sea el uso que alguien
le quiera dar.

\section{Conclusiones sobre la aportación de SWAML}

FIXME(SIOC, etc)

\section{Trabajo relacionado}

FIXME(¿la idea de swaml a que más podría aplicarse?)

\section{Conclusiones sobre el software utilizado}

Ha sido mucha la diversidad de software (complidores, bibliotecas y herramientas)
utilizado para la realización de este proyecto. Nótese que todo él se ha podido
desarrollar utilizando únicamente \emph{software libre}.

\subsection{Python}

FIXME

\subsection{Bibliotecas\label{sec:conclu:bib}}

FIXME(¿más?)

\subsubsection{RDFLib}

RDFLib\footnote{\url{http://rdflib.net/}} FIXME

\subsubsection{dom.xml}

dom.xml\footnote{\url{http://docs.python.org/lib/module-xml.dom.html}} FIXME

\subsubsection{PyGTK}

FIXME

\subsection{Herramientas}

FIXME

\subsubsection{Subversion}

FIXME

\subsubsection{Autotools}

FIXME

\subsubsection{PyDev}

PyDev es un plugin para Eclipse FIXME

\subsubsection{Ant}

FIXME

\subsubsection{Gazpacho}

FIXME

\subsubsection{SWOOP}

Existen varios editores libres para ontologías OWL:

\begin{itemize}
  \item Protégé\footnote{\url{http://protege.stanford.edu/plugins/owl/}}
  \item SWeDE\footnote{\url{http://owl-eclipse.projects.semwebcentral.org/}}
  \item SWOOP\footnote{\url{http://www.mindswap.org/2004/SWOOP/}}
\end{itemize}

Los dos primeros no son más que plug-ins para dar soporte a OWL en dos 
frameworks. Y el tercero es un editor pensado y desarrollado explicitamente
para trabajar con OWL.

Después de las pruebas realizadas, SWOOP resultó ser una herramienta más sencilla,
comoda de usar y potente que las otras dos.

\begin{figure}[ht]
	\centering
	\includegraphics[width=11cm]{images/swoop.png}
	\caption{SWOOP editando la ontología de SWAML en OWL DL}
	\label{fig:evoWeb}
\end{figure}

Alguna de las caracteristicas más interesantes de SWOOP son:

\begin{itemize}
  \item Interfaz de usuario hypermedia similar a la de un navegador convencional, 
	con elementos (pestañas, marcadores, etc) que hacen la interfaz más 
	amigable.
  \item Soporte de depuración de la ontología.
  \item Cliente para hacer razonamientos sencillos con Pellet.
  \item FIXME
\end{itemize}

Un problema común en todas estas herramientas de alto nivel para trabajar con
grafos RDF es el serializado del grafo a sintáxis XML. No por su corrección,
que la herramienta lo hace perfectamente, sino por su orden: es muy difícil
que al serializar queden todos los nodos en el mismo orden. Por tanto es muy
difícil conocer las diferencias entre distintas versiones con las herramientas
convencionales (principalmente \texttt{diff}).

\subsubsection{phpWiki}

FIXME

\subsubsection{Google Maps}

FIXME(KML)

\subsubsection{\LaTeX}

\TeX/\LaTeX FIXME

\paragraph{Kile}

\paragraph{JabRef}

\section{Lineas de futuro}

El trabajo realizado hasta ahora con SWAML no ha hecho sino empezar, abriendo las 
puertas hacia una linea de desarrollo que puede significar un foco importante
en cuanto a la publicación en formatos semánticos de todo el conocimiento contenido
en esas miles de listas de correo existentes por lo largo y ancho de la internet
actual.

Aunque puede que no se incluyan todas las existentes, estas on algunas de las lineas
que seria interesante el proyecto abarcara algún día:

\subsection*{Integración con Mailman}

FIXME

\subsection*{API para DIG}

Sería interesante disponer de una API en Python para utilizar razonadores DL que implementen 
el protocolo DIG\footnote{\url{http://dig.cs.manchester.ac.uk/}}. 

Existen ya API's similares en Java (dentro del framework
Jena\footnote{\url{http://jena.sourceforge.net/how-to/dig-reasoner.html}} o incluso en Haskell 
(implementada dentro del proyecto WESO\footnote{\url{http://weso.sourceforge.net/}}. Por eso 
sería interesante implementar el protocolo DIG en Python, y no sería una tarea
en la que nos encontraríamos en solitario\cite{PythonOWL}.

FIXME



\appendix


\chapter{Anexos}

\section{Ontolog�a}

Se ha desarrollado, teniendo muy en cuenta las buenas pr�cticas que recomienda el W3C para publicar vocabularios
RDF\footnote{http://www.w3.org/TR/swbp-vocab-pub/}, una sencilla ontolog�a que formalizara las representaci�n 
de una lista de correo que SWAML publica en RDF. 

La URI para la ontolog�a es \url{http://swaml.berlios.de/ns/swaml.owl}, siendo adem�s esa URI la URL donde se 
encuentra la ontolog�a en OWL DL desarrollada.

La ontolog�a define tres clases para representar una lista de correo:

\begin{figure}[h]
	\centering
	\includegraphics[width=12cm]{images/swaml-owl.png}
	\caption{Representaci�n gr�fica de las clases descritas en la ontolog�a de SWAML}
	\label{fig:evoWeb}
\end{figure}

FIXME(incluir swaml.owl)

\section{C�digo fuente\label{sec:source}} 

FIXME(automatizar)

\newpage

\section{Licencias} 

FIXME(ojear \url{http://sciencecommons.org/literature/scholars_copyright})

\subsection{Creative Commons Reconocimiento-CompartirIgual 2.5\label{sec:license.cc}}


\begin{center}
  {\Large \sc Creative Commons
  \\\vspace{3mm}Reconocimiento 2.5 Espa�a}
  \\\vspace{3mm}\url{http://creativecommons.org/licenses/by/2.5/es/}
\end{center}



Usted es libre de:
\begin{itemize}
  \item copiar, distribuir y comunicar p�blicamente la obra
  \item hacer obras derivadas
  \item hacer un uso comercial de esta obra
\end{itemize}

Bajo las condiciones siguientes:
\begin{itemize}
  \item Reconocimiento. Debe reconocer los cr�ditos de la obra de la manera 
	especificada por el autor o el licenciador.
\end{itemize}

\begin{itemize}
  \item Al reutilizar o distribuir la obra, tiene que dejar bien claro los t�rminos 
	de la licencia de esta obra.
  \item Alguna de estas condiciones puede no aplicarse si se obtiene el permiso del
	titular de los derechos de autor
\end{itemize}

Los derechos derivados de usos leg�timos u otras limitaciones reconocidas por ley no 
se ven afectados por lo anterior.

\newpage

\subsection{GNU General Public License (GPL)\label{sec:license.gpl}}

\input{licenses/gpl-2.0.tex}

\newpage

\section{Acerca}

Este documento ha sido escrito usando \LaTeX.


%FIXME: listado de enlaces \url{}

\bibliography{bibliografia}

\end{document}

