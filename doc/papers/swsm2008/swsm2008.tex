% This is "www2008-sample.tex" copied from "www2005-sample.tex" V1.2 January 26 2004
% This file should be compiled with V1.4 of "www2008-submission.class"
%
% This example file demonstrates the use of the 'www2008-submission.cls'
% V1.4 LaTeX2e document class file. It is for those submitting
% articles to the WWW'04 Conference WHO DO NOT WISH TO 
% STRICTLY ADHERE TO THE SIGS (PUBS-BOARD-ENDORSED) STYLE.
% The 'www2008-submission.cls' file will produce a similar-looking,
% albeit, 'tighter' paper resulting in, invariably, fewer pages.
%
% ----------------------------------------------------------------------------------------------------------------
% This .tex file (and associated .cls V1.4) produces:
%       1) NO Permission Statement
%       2) WWW'04-specific conference (location) information
%       3) The Copyright Line with ACM data
%       4) NO page numbers
%
% ---------------------------------------------------------------------------------------------------------------
% This .tex source is an example which *does* use
% the .bib file (from which the .bbl file % is produced).
% REMEMBER HOWEVER: After having produced the .bbl file,
% and prior to final submission, you *NEED* to 'insert'
% your .bbl file into your source .tex file so as to provide
% ONE 'self-contained' source file.
%
% ================= IF YOU HAVE QUESTIONS =======================
% Questions regarding the SIGS styles, SIGS policies and
% procedures, Conferences etc. should be sent to
% Julie Goetz (goetz@acm.org) or Adrienne Griscti (griscti@acm.org)
%
% Technical questions only to
% Gerald Murray (murray@acm.org)
% ===============================================================
%
% For tracking purposes - this is V1.2 - January 26 2004
\documentclass{../templates/www2008-submission}

\begin{document}
%
\title{Semantic Mining of Social Online Communities}
%\subtitle{[Extended Abstract]
%\titlenote{A full version of this paper is available as
%\textit{Author's Guide to Preparing ACM SIG Proceedings Using
%\LaTeX$2_\epsilon$\ and BibTeX} at
%\texttt{www.acm.org/eaddress.htm}}}
%
% You need the command \numberofauthors to handle the "boxing"
% and alignment of the authors under the title, and to add
% a section for authors number 4 through n.
%
% Up to the first three authors are aligned under the title;
% use the \alignauthor commands below to handle those names
% and affiliations. Add names, affiliations, addresses for
% additional authors as the argument to \additionalauthors;
% these will be set for you without further effort on your
% part as the last section in the body of your article BEFORE
% References or any Appendices.

\numberofauthors{4}
%
% Put no more than the first THREE authors in the \author command

% NOTE: All authors should be on the first page. For instructions
% for more than 3 authors, see:
% http://www.acm.org/sigs/pubs/proceed/sigfaq.htm#a18

\author{
%
% The command \alignauthor (no curly braces needed) should
% precede each author name, affiliation/snail-mail address and
% e-mail address. Additionally, tag each line of
% affiliation/address with \affaddr, and tag the
%% e-mail address with \email.
\alignauthor Diego Berrueta\\
       \affaddr{Fundaci\'on CTIC}\\
       \affaddr{Gij\'on, Asturias, Spain}\\
       \email{diego.berrueta@fundacionctic.org}
\alignauthor Luis Polo\\
       \affaddr{Fundaci\'on CTIC}\\
       \affaddr{Gij\'on, Asturias, Spain}\\
       \email{luis.polo@fundacionctic.org}
\and
\alignauthor Sergio Fern\'andez\\
       \affaddr{Fundaci\'on CTIC}\\
       \affaddr{Gij\'on, Asturias, Spain}\\
       \email{sergio.fernandez@fundacionctic.org}
\alignauthor Lian Shi\\
       \affaddr{Fundaci\'on CTIC}\\
       \affaddr{Gij\'on, Asturias, Spain}\\
       \email{lian.shi@fundacionctic.org}
}

%\additionalauthors{Additional authors: John Smith (The Th{\o}rv\"{a}ld Group,
%email: {\texttt{jsmith@affiliation.org}}) and Julius P.~Kumquat
%(The Kumquat Consortium, email: {\texttt{jpkumquat@consortium.net}}).}

%\date{5 February 2008}

\maketitle

\begin{abstract}
The Online Communities are a very rich font of collective 
knowledge that are untapped. In this paper we present 
several approaches to mining this kind of communities to 
allow a semantic use of all this  information.
FIXME: completar y lucir
\end{abstract}

% A category with only the three required fields
%\category{H.4.m}{Information Systems}{Miscellaneous}
%\category{D.2}{Software}{Software Engineering}
%A category including the fourth, optional field follows...
%\category{D.2.8}{Software Engineering}{Metrics}[complexity measures,
%performance measures]

\keywords{semantic mining, rdf, xsl}

\section{Introduction}

FIXME

\section{Technologies}

FIXME: intro SIOC\cite{Breslin2005}, etc

\section{Approaches}

There are a lot of possible approaches that could be used to
extract the information published in social online communities.
But mainly we centered in two kind: intrusive
and unintrusive techniques.

FIXME


\subsection{Intrusive techniques}

FIXME

\subsection{Unntrusive techniques}

FIXME

\section{Common problems}

FIXME

\section{Experimentation with Free Sofware communities}

FIXME

\section{\label{sec:conclusions}Conclusions}

FIXME

\section*{Acknowledgements}

FIXME


% The following two commands are all you need in the
% initial runs of your .tex file to
% produce the bibliography for the citations in your paper.
\bibliographystyle{abbrv}
\bibliography{../references} 

\balancecolumns

\end{document}
