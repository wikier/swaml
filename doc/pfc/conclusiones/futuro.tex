
\section{Lineas de futuro}

El trabajo realizado hasta ahora con SWAML no ha hecho sino empezar, abriendo las 
puertas hacia una linea de desarrollo que puede significar un foco importante
en cuanto a la publicación en formatos semánticos de todo el conocimiento contenido
en esas miles de listas de correo existentes por lo largo y ancho de la internet
actual.

Aunque puede que no se incluyan todas las existentes, estas son algunas de las lineas
que seria interesante el proyecto abarcara algún día:

\subsection*{Marcado semántico para el cuerpo de los mensajes}

La información semántica de un lista de correo publicada por SWAML sólo tiene una
laguna: el marcado semántico del cuerpo de mensaje (\texttt{sioc:content}). En
la actualidad no se realiza ningún tipo de procesamiento al contenido de ese campo.
Quizás fuera interesante, bien de forma manual o automática, mejorar el marcado
semántico de ese contenido para así poder explotar de manera más eficaz esa 
información.

\subsection*{API en Python para SIOC}

Ya hay disponible uno similar en PHP\footnote{\url{http://sioc-project.org/phpapi}}. 
Abstraer todo el código de SWAML relacionado con SIOC a un API independiente permitiría,
además de mejorar el diseño de esa parte del proyecto, proveer a otras aplicaciones 
en Python (o con bindings para Python) la posibilidad de exportar a SIOC los datos 
que manejan.

\subsection*{Integración con Mailman}

GNU Mailman\footnote{\url{http://www.gnu.org/software/mailman/index.html}} es quizás el
sistema de gestión de listas de correo más usado y extendido en la actualidad. Desarrollado
también en Python, una integración de SWAML con Mailman supondría que incontables listas
de correo repartidas por todo el mundo publicarían la descripción semántica de sus
archivos antiguos. Indudablemente sería un hecho muy relevante para el proyecto, pero
se trata de objetivo difícil de alcanzar por la calidad del software requerido en
proyectos de tal envergadura.

\subsection*{API para DIG}

Sería interesante disponer de un API en Python para utilizar razonadores DL que implementen 
el protocolo DIG\footnote{\url{http://dig.cs.manchester.ac.uk/}}. 

Existen ya API's similares en Java (dentro del framework
Jena\footnote{\url{http://jena.sourceforge.net/how-to/dig-reasoner.html}}) o incluso en Haskell 
(implementada dentro del proyecto WESO\footnote{\url{http://weso.sourceforge.net/}}). Por eso 
sería interesante implementar el protocolo DIG en Python, y no sería una tarea
en la que nos encontraríamos en solitario\cite{PythonOWL}.

