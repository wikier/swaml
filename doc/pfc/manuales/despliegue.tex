
\section{Manual de despliegue}

\subsection*{Requisitos técnicos}

El proyecto exige ciertos requisitos técnicos para su correcto
funcionamiento.

\subsubsection*{Intérprete de Python}

Evidentemente será necesario disponer en nuestro sistema de algún intérprete
de Python 2.4 (Python, IronPython u otros...). Se puede 
obtener\footnote{\url{http://www.python.org/download/}} de la propia página 
oficial del lenguaje, aunque se encuentra empaquetada para múltiples sistemas 
operativos. En Debian GNU/Linux\footnote{\url{http://www.debian.org/}}, por 
ejemplo, bastaría con hacer:

\begin{center}
	\texttt{apt-get install python2.4}
\end{center}

\subsubsection*{Bibliotecas necesarias}

Como se puede ver en la sección~\ref{sec:conclu:bib}, han sido varias las
bibliotecas utilizadas en el proyecto que debemos tener instaladas:

\begin{itemize}
  \item RDFLib\footnote{\url{http://rdflib.net/}} = 2.3.1
  \item PyXML\footnote{\url{http://pyxml.sourceforge.net/}}
  \item GTK+\footnote{\url{http://www.gtk.org/}} >= 2.6.0
  \item PyGTK\footnote{\url{http://www.pygtk.org/}} >= 2.6.0
  \item gazpacho\footnote{\url{http://gazpacho.sicem.biz/}} >= 0.6.6
\end{itemize}

La forma de instalarlas ya dependerá del sistema operativo que utilice.

\subsection*{Descomprimir SWAML}

SWAML se distribuye\footnote{\url{http://swaml.berlios.de/files}} comprimido
en ficheros \texttt{.tar.gz}  que podrá descomprimir con casi cualquier 
software de descompresión de ficheros (gzip, tar, WinZip, WinRAR, etc). En 
los propios tarballs\footnote{Nombre por el que conoce los ficheros \texttt{.tar.gz}}
se distribuye esta misma documentación (en un fichero llamado \texttt{INSTALL})
de forma un poco más abreviada.

\subsection*{Instalar SWAML}

SWAML puede ser usado perfectamente si necesidad de instalarse. Aún así la instalación 
de SWAML se realiza con una simple regla del Makefile:

\begin{center}
	\texttt{make install}
\end{center}

\subsection*{Desinstalar SWAML}

También para desinstalarlo basta invocar una simple regla de Makefile:

\begin{center}
	\texttt{make uninstall}
\end{center}
