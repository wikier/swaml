
\subsection{Plan de pruebas}

\subsubsection{Pruebas de requisitos funcionales}

A continuación de presentan las pruebas que se deberán realizar para comprobar 
el correcto funcionamiento de la aplicación con respecto a cada uno de los 
casos de uso presentados en la sección~\ref{sec:espec-casos-uso} bajo el
epígrafe \emph{«Especificación de los Casos de Uso»}. El conjunto de pruebas 
aquí presentadas se corresponde con las denominadas pruebas de caja negra. 

\begin{itemize}

  \item \textbf{Parametrizar el sistema:}
	\begin{table}[H]
	 \begin{center}
	  \begin{tabular}{|p{5cm}|p{7cm}|}
		\hline
		\textbf{Prueba} & \textbf{Resultado esperado} \\ 
		\hline
		El usuario edita a mano el fichero 
		& 
		Impredecible
		\\\hline
		Se genera una configuración utilizando el asistente
		& 
		Configuración correctamente formada en el fichero indicado
		\\\hline
	  \end{tabular}
	  \caption{Plan para las pruebas de «parametrizar el sistema»}
	 \end{center}
	\end{table}

  \item \textbf{Publicar}
	\begin{table}[H]
	 \begin{center}
	  \begin{tabular}{|p{5cm}|p{7cm}|}
		\hline
		\textbf{Prueba} & \textbf{Resultado esperado} \\ 
		\hline
		Mailbox sin ninguna ninguna particularidad 
		& 
		Se genera correctamente su representación en RDF
		\\\hline
		El mailbox no existe en la ruta indicada 
		& 
		Se advierte al usuario de dicho error
		\\\hline
		Mailbox de gran extensión (más de 9.000 mensajes)
		& 
		Se genera correctamente su representación en RDF
		\\\hline
		Mailbox con correos repetidos
		& 
		Se genera correctamente su representación en RDF
		\\\hline
	  \end{tabular}
	  \caption{Plan para las pruebas de «publicar»}
	 \end{center}
	\end{table}

  \item \textbf{Enriquecer los datos}
	\begin{table}[H]
	 \begin{center}
	  \begin{tabular}{|p{5cm}|p{7cm}|}
		\hline
		\textbf{Prueba} & \textbf{Resultado esperado} \\ 
		\hline
		Se dispone del FOAF de algunos suscriptores
		& 
		Se enriquece la información de esos suscriptores
		\\\hline
		No se dispone de ficheros FOAF asociado a ningún suscriptor
		& 
		La lista no se enriquece
		\\\hline
		El suscriptor tiene más de un FOAF
		& 
		Se enriquece con el FOAF que se reciba del servicio externo
		\\\hline
	  \end{tabular}
	  \caption{Plan para las pruebas de «enriquecer datos»}
	 \end{center}
	\end{table}

  \item \textbf{Consular los archivos generados}
	\begin{table}[H]
	 \begin{center}
	  \begin{tabular}{|p{5cm}|p{7cm}|}
		\hline
		\textbf{Prueba} & \textbf{Resultado esperado} \\ 
		\hline
		Se encuentra la información consultada por el usuario
		& 
		Se le devuelve esa información
		\\\hline
		No se encuentra la información consultada por el usuario
		& 
		Se advierte al usuario de que su consulta no ha tenido resultados
		\\\hline
	  \end{tabular}
	  \caption{Plan para las pruebas de «consultar»}
	 \end{center}
	\end{table}

  \item \textbf{Consultar la información extra generada}
	\begin{table}[H]
	 \begin{center}
	  \begin{tabular}{|p{5cm}|p{7cm}|}
		\hline
		\textbf{Prueba} & \textbf{Resultado esperado} \\ 
		\hline
		Hay meta-información generada
		& 
		Se puede explotar esa información con las aplicaciones disponibles
		\\\hline
		No hay meta-información asociada
		& 
		La aplicación que la explote deberá controlar esta situación
		\\\hline
	  \end{tabular}
	  \caption{Plan para las pruebas de «consultar información extra»}
	 \end{center}
	\end{table}

\end{itemize}


\subsubsection{Pruebas de requisitos no funcionales}

Se deberán realizar, al menos, las siguientes pruebas:

\begin{itemize}
  \item Comprobar el correcto funcionamiento del todos los servicios en GNU/Linux,
	Microsoft Windows y Apple MacOS.
  \item Comprobar si los requisitos mínimos del ordenador son válidos.
\end{itemize}
