
\section{Análisis y diseño (SAD)}

La arquitectura del software no es algo unidimensional, sino que esta formado 
por múltiples vistas, como se puede ver en la figura~\ref{fig:sad}, con el 
fin de detallar la funcionalidad, organización y topología del sistema.

\begin{figure}[hp]
	\centering
	\includegraphics[width=10cm]{images/sad.png}
	\caption{Vista de la arquitectura software}
	\label{fig:sad}
\end{figure}

\begin{itemize}
  \item \textbf{Vista de casos de uso:} amplía la representación gráfica 
	de los escenarios descritos durante la especificaciones de casos 
	de uso, definiendo así el comportamiento del sistema en términos 
	funcionales.
  \item \textbf{Vista lógica:} se muestran los requisitos funcionales 
	del sistema. La arquitectura lógica se captura en diagramas de clases 
	que contienen clases y relaciones, para representar las abstracciones 
	clave del sistema en desarrollo.
  \item \textbf{Vista de proceso:} muestra las interacciones en tiempo real de
	los distintos componentes del sistema software, teniendo en cuenta
	requisitos no tenidos en cuenta en otras fases: el rendimiento, la 
	fiabilidad, escalabilidad, integridad, organización del sistema y 
	sincronización. 
  \item \textbf{Vista de implementación:} descompone el sistema software en
	\emph{paquetes}, teniendo en cuenta aspectos como la organización del 
	software, la reutilización, modularidad, facilidad de desarrollo y 
	restricciones impuestas por los lenguajes de programación y las 
	herramientas usadas en el desarrollo
  \item \textbf{Vista de distribución:} muestra la topología de la arquitectura
	de manera que sea más comprensible por el equipo de desarrollo.
\end{itemize}


\subsection{Vista de casos de uso}

FIXME



\subsection{Vista lógica}

\subsubsection{Diagrama de clase de SWAML}

\begin{figure}[p]
	\centering
 	\includegraphics[width=14cm]{images/uml/clases/swaml.png}
	\caption{Diagrama de clase de SWAML}
	\label{fig:uml:swaml}
\end{figure}

\subsubsection{Diagrama de clase de Buxon}

\begin{figure}[p]
	\centering
 	\includegraphics[width=15cm]{images/uml/clases/buxon.png}
	\caption{Diagrama de clase de Buxon}
	\label{fig:uml:buxon}
\end{figure}



\subsection{Vista de proceso}

Se muestran los componentes ejecutables que funcionan en el sistema en 
tiempo de ejecución. El análisis y diseño realizado han dado lugar a 
varios componentes (SWAML, configWizard, Buxon, FOAF Enricher y 
KML Exporter) que interactuan según el diagrama de componentes descrito 
en la figura~\ref{fig:uml:componentes}.

\begin{figure}[H]
	\centering
	\includegraphics[width=15cm]{images/uml/componentes.png}
	\caption{Diagrama de componentes}
	\label{fig:uml:componentes}
\end{figure}

\subsubsection{SWAML}

Representa el núcleo principal de la aplicación, formado por el script
\texttt{swaml.py}. Implementa toda la lógica de la aplicación para
exportar un mailbox a RDF.

\subsubsection{configWizard}

Mediante el script \texttt{configWizard.py} se provee una forma ágil y 
sencilla de crear los ficheros de configuración que debe recibir SWAML 
como entrada.

\subsubsection{FOAF Enricher}

Representa el medio para enriquecer la información de los suscriptores
usando sus ficheros FOAF con fuente primaria de información.

\subsubsection{KML Exporter}

Mediante este componente se obtiene la información geográfica de los
distintos suscriptores de una lista de correo.

\subsubsection{Buxon}

Representa la interfaz de usuario para visualizar listas de correo
exportadas en SIOC.






\subsection{Vista de implementación}

Se describe la estructura del modelo de implementación, que está totalmente 
relacionado con la herramienta de programación que se utilice para realizar 
la aplicación. En Python las clases de agrupan en paquetes (\textit{package}).

SWAML se ha dividido en dos paquetes, \texttt{swaml} y \texttt{swaml.classes},
tal y como se puede ver en el diagrama de componentes de la 
figura~\ref{fig:uml:implementación}.

\begin{figure}[H]
	\centering
	\includegraphics[width=8cm]{images/uml/implementacion.png}
	\caption{Diagrama de implementación}
	\label{fig:uml:implementación}
\end{figure}

En la documentación en linea\footnote{\url{http://swaml.berlios.de/doc/}}
puede obtenerse con más detalle la composición de cada paquete.


\subsection{Vista de distribución}

A partir de los componentes definidos en la vista de proceso,
se pueden extraer dos componentes principales importantes a la
hora de desplegar la aplicación.

\begin{figure}[H]
	\centering
	\includegraphics[width=10cm]{images/uml/despliegue.png}
	\caption{Diagrama de despliegue}
	\label{fig:uml:despliegue}
\end{figure}

Como se puede ver en el diagrama de despliegue descrito en la 
figura~\ref{fig:uml:despliegue}, el proceso principal de SWAML 
deberá ejecutarse en un servidor con capacidad para servir 
ficheros por HTTP. Buxon podrá, o no, estar en otro PC cliente
siempre y cuanto pueda establecer una conexión HTTP con el servidor
sobre el que han sido publicados los datos exportados por SWAML.


