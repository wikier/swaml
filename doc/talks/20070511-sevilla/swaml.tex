\documentclass[spanish,notes=hide]{beamer}

%Para crear una versión 'handout' (Copyright: Diego Berrueta)
%\documentclass[handout,notes=show]{beamer}

\usetheme{Marburg}

\usepackage[spanish]{babel}
\usepackage[utf8]{inputenc}
\usepackage{listings}
\usepackage{graphicx}
\usepackage{colortbl}
\usepackage{array}
\usepackage{eurosym}

\title{SWAML}
\subtitle{Semantic Web Archive of Mailing Lists}
\author{Sergio Fern\'andez}
\institute{%
        \email{sergio@wikier.org}\\
	\vspace{1cm}
	E.U. de Ingenier\'ia T\'ecnica en Inform\'atica de Oviedo\\
	(Universidad de Oviedo)\\
	\vspace{1cm}
	\href{http://swaml.berlios.de/}{\begin{Large}\textbf{http://swaml.berlios.de/}\end{Large}}\\
        \vspace{0.5cm}
}
\date{11 de Mayo de 2007}

\begin{document}


%listings

\definecolor{darkred}{rgb}{0.5, 0, 0}
\definecolor{violet}{rgb}{1, 0, 1}
\definecolor{green}{rgb}{0.3, 0.95, 0.3}
\definecolor{listinggray}{gray}{0.97}

\lstset{
	basewidth=0.50em,
	backgroundcolor=\color{listinggray},
	basicstyle=\footnotesize\ttfamily,
	keywordstyle=\bfseries,
	stringstyle=\itshape,
	commentstyle=\itshape,
	showspaces=false,
	showtabs=false,
	showstringspaces=false,
	frame=trbl,
	extendedchars=true,
	numbers=none,
	aboveskip=0.5cm,
	belowskip=0.5cm,
	xleftmargin=0cm,
	xrightmargin=0cm
}

\lstdefinelanguage{mbox}{%no funciona!
	morekeywords = {From, Message, Date, Organization, To, Subject }
}

\defverbatim[colored]\MBOX{
\lstset{language=mbox}
\begin{lstlisting}
...
From sioc-dev@googlegroups.com Fri Sep 15 13:35:44 2006
Message-ID: <1158352519.450b0e871c79e@courrier.privatedns.com>
Date: Fri, 15 Sep 2006 16:35:19 -0400
From: Frederick Giasson <fred@fgiasson.com>
To: sioc-dev@googlegroups.com
Subject: Implementation of the SIOC v1.08 ontology in Talk Digger
...
From sioc-dev@googlegroups.com Tue Sep 19 07:10:22 2006
From: Kjetil Kjernsmo <kjetilk@opera.com>
Organization: Opera Software ASA
To: sioc-dev@googlegroups.com
Subject: Re: User vs. Person complexity
Date: Tue, 19 Sep 2006 16:09:15 +0200
...
\end{lstlisting}
}


\maketitle

\section{Introducción}
\frame
{
  \frametitle{Introducción}

  \begin{itemize}
   \item<2-> \begin{large}\textbf{Situación actual:}\end{large}
	\begin{itemize}
	  \item \begin{large}miles de listas de correo\end{large}
	  \item \begin{large}publicación (sintáctica) en (X)HTML\end{large}
	\end{itemize}
   \vspace{1cm}
   \item<3-> \textbf{Problemas:}
	\begin{itemize}
	  \item \begin{large}pérdida de información\end{large}
	  \item \begin{large}marcado sin ningún valor semántico\end{large}
	  \item \begin{large}problemas usando motores de búsqueda convencionales\end{large}
	\end{itemize}
  \end{itemize}
}
\frame
{
  FIXME: captura de pantalla con una busqueda en google
}

\section{La Web semántica}
\frame
{
  \frametitle{La Web semántica}

  \begin{Large}
    La Web semántica es una \textbf{extensión} sobre la Web actual donde el 
    contentido Web debe ser expresado de forma que sea entendido, interpretado 
    y usado por máquinas.
  \end{Large}
  
  \vspace{1cm}

  \begin{Large}
    Proviene de la visión de Tim Berners-Lee (Director del W3C) de la Web como un
    medio \textbf{universal} para el intercambio de \textbf{datos}, \textbf{información} 
    y \textbf{conocimiento}.
  \end{Large}
}
\frame
{
  \frametitle{La pila de la Web semántica}

  \begin{center}
    \includegraphics[width=0.8\textwidth]{images/semantic-web-stack.png}
  \end{center}

}
\frame
{
  \frametitle{RDF}

  \begin{Large}
     \textbf{R}esource \textbf{D}escription \textbf{F}ramework es una familia de
     tecnologías (RDF, SPARQL, etc...) de W3C para modelar conocimiento. 
  \end{Large}

  \vspace{0.7cm}

  \begin{Large}
     RDF se basa en un modelo de tripletas:
     \begin{center}
       \includegraphics[width=0.8\textwidth]{images/triple.png}
     \end{center}
     El \textbf{sujeto} es un recurso, identificado por un URI 
     (\textit{Uniform Resource Identifier}), que se relaciona mediante un 
     \textbf{predicado} con un \textbf{objeto}, que puede ser a su vez otro 
     recurso o un literal.
  \end{Large}
}
\frame
{
  \frametitle{SIOC}

  \begin{Large}
    \textbf{S}emantically-\textbf{I}nterlinked \textbf{O}nline \textbf{C}ommunities 
    is an ontology that provides a vocabulary to interconnect different discussion 
    methods such as blogs, web-based forums and mailing lists.
  \end{Large}

  \begin{center}
    \includegraphics[width=0.8\textwidth]{images/sioc-terms.png}
  \end{center}


}
\frame{
  \frametitle{Extending SIOC Ontology}

  \begin{Large}
     SIOC is an almost perfect match for our purpose, using instances of
     \texttt{sioc:Forum}, \texttt{sioc:Post} and \texttt{sioc:User} classes.
  \end{Large}

  \vspace{0.7cm}

  \begin{Large}
    However, additional object properties were required in order to retain the 
    sequence of messages published in a mailing list. Thus, we extended the SIOC 
    ontology with \textbf{two new properties}.
  \end{Large}

}

\section{Software tools}
\frame
{
  \frametitle{Software tools}

  We built two software tools as part of this project:
  \vspace{0.5cm}
  \begin{itemize}
    \item<2->	\begin{Large}\textbf{SWAM}L is a non-interactive, 
		command-line application whose main purpose is to 
		translate mailboxes into sioc:Forum instances in 
		RDF.\end{Large}
    \vspace{0.5cm}
    \item<3->	\begin{Large}\textbf{Buxon} is a graphical browser 
		for \texttt{sioc:Forum} instances.\end{Large}
  \end{itemize}
}
\frame
{
  \frametitle{Software tools}

  \begin{center}
    \includegraphics[width=0.85\textwidth]{images/swaml-tools.png}
  \end{center}
}
\frame
{
  \frametitle{SWAML}

  \begin{columns}
   \begin{column}{0.5\textwidth}
	\begin{center}
	  \only<1>{\includegraphics[width=0.95\textwidth]{images/swaml-0.png}}
	  \only<2>{\includegraphics[width=0.95\textwidth]{images/swaml-1.png}}
	  \only<3>{\includegraphics[width=0.95\textwidth]{images/swaml-2.png}}
	  \only<4>{\includegraphics[width=0.95\textwidth]{images/swaml-3.png}}
	\end{center}
   \end{column}
   \begin{column}{0.5\textwidth}
	\begin{Large}batch process:\end{Large}
	\begin{enumerate}
	 \item<2-> mbox
	 \item<3-> parser
	 \item<4-> serialize to RDF/XML 
	\end{enumerate}
   \end{column}
  \end{columns}
}
\frame
{
  \frametitle{SWAML and KML}

  SWAML also generates a KML file that contains the geographical coordinates 
  of the mailing list subscribers.

  \begin{center}
    \includegraphics[width=0.7\textwidth]{images/googlemaps.png}
  \end{center}
}
\frame
{
  \frametitle{Buxon}

  \begin{columns}
   \begin{column}{0.32\textwidth}
	\begin{itemize}
	  \item a \texttt{sioc:Forum} browser
	  \item it recomposes mailing lists' threads
	  \item one of the most important implementation of SIOC
	\end{itemize}
   \end{column}
   \begin{column}{0.68\textwidth}
	\includegraphics[width=\textwidth]{images/buxon.png}
   \end{column}
  \end{columns}
}

\section{Conclusions and future work}
\frame
{
  \frametitle{Conclusions}

  \begin{Large}
    At the time of this talk e-mail was almost excluded from the Semantic Web.

    \vspace{1cm}

    This project, in combinationwith the generic SIOC framework, fulfills a 
    much-needed requirement for the Semantic Web: \textbf{to be able to refer 
    to semantic versions of e-mail messages and their properties using resource 
    URIs.}
    
  \end{Large}
}
\frame
{
  \frametitle{Future Work}

  \begin{itemize}
   \item \begin{Large}Integration of the SWAML process with popular HTML-based mailing list archivers.\end{Large}
   \item \begin{Large}Integration could be pushed further away through RDFa.\end{Large}
   \item \begin{Large}Semantic annotation relative to the meaning of the messages.\end{Large}
   \item \begin{Large}A simple extension to SWAML that makes it possible to read the contents of a GMail account has 
		been developed.\end{Large}
   \item \begin{Large}SIOC submission to W3C.\end{Large}
  \end{itemize}
}

\section{Demostración}
\frame
{
  \frametitle{Demostración práctica}

  \begin{center}
    \LARGE{\textbf{Let's go...}}
  \end{center}
}

\appendix

\section{Agradecimientos}
\frame
{
  \frametitle{Agradecimientos}
  
  \begin{Large}
    The authors would like to express their gratitude to Dr. John Breslin and 
    Uldis Bojars from DERI Galway, whose support and contributions have been 
    of great help to this project. 
  \end{Large}

  \vspace{1cm}

  \begin{Large}
    Also to Ignacio Barrientos by his contribution packaging the project for 
    Debian GNU/Linux.
  \end{Large}
}

\section{Más información}
\frame
{
  \begin{center}
    Más información en la página Web del proyecto:\\
    \vspace{1cm}
    \LARGE{\href{http://swaml.berlios.de/}{http://swaml.berlios.de/}}\\
  \end{center}

}

\section{Licencia}
\frame
{
  \begin{center}
    \LARGE{%
	\textbf{SWAML}\\
	\textbf{Semantic Web Archive\\of Mailing Lists}
    }\\
    \vspace{1cm}
    \vspace{1cm}
    \begin{tiny}
	Esta charla se distribuye bajo los términos de la licencia:\\
	\textbf{CreativeCommons}\\
	\includegraphics[width=3.5cm]{images/creativecommons.png}
    \end{tiny}
  \end{center}
}

\end{document}
