\documentclass[spanish,notes=hide]{beamer}

%Para crear una versión 'handout' (Copyright: Diego Berrueta)
%\documentclass[handout,notes=show]{beamer}

\usetheme{Warsaw}
\usepackage{beamerthemesplit}

\usepackage[spanish]{babel}
\usepackage[utf8]{inputenc}
\usepackage{listings}
\usepackage{graphicx}
\usepackage{colortbl}

\title{SWAML}
\subtitle{Publicaci\'on de listas de correo en Web Sem\'antica}
\author{Sergio Fern\'andez L\'opez}
\institute{%
	\href{http://swaml.berlios.de/}{http://swaml.berlios.de/}\\
	\vspace{0.7cm}
	Proyecto Fin de Carrera\\
	E.U. de Ingenier\'ia T\'ecnica en Inform\'atica de Oviedo
}
\date{20 de Diciembre de 2006}

\begin{document}

\frame{
  \note[item]{Saludar}
  \note[item]{Presentarse}
  \note[item]{Con su permiso procederé a presentarles mi proyecto fin de carrera 
    \textit{SWAML, publicación de listas de correo en Web Semántica}}

  \titlepage
}

%\frame{\tableofcontents}

\section{Introducción}

\subsection{Situación actual}
\frame
{
  \frametitle{Panorama actual}

  %FIXME: abreviar estos parrafos

  \begin{itemize}
   \item<2-> Situación:
	\begin{itemize}
	  \item Miles de listas de correo de las más variopinta temática
	  \item Publicación en HTML de los archivos antiguos %FIXME: captura de una lista de W3C
	\end{itemize}
   \item<3-> Problemas:
	\begin{itemize}
	  \item Pérdida de toda posibilidad de recuperar esa información
	  \item Marcado estructurado sin valor semántico (problemas de internacionalización 
		y accesibilidad entre otros)
	  \item Duplicidad de datos en los motores de búsqueda tradicionales
	\end{itemize}
  \end{itemize}
}

\subsection{Objetivos}
\frame
{
  \frametitle{Objetivos}

  \begin{itemize}
   \item<1-> Objetivo principal: 
     \begin{itemize}
      \item \textbf{Publicación de los archivos antiguos de listas de correo en un formato rico semánticamente.}
     \end{itemize}
   \item<2-> Varios objetivos secundarios:
	\begin{itemize}
	  \item Maximizar la reutilización de la infraestructura disponible previamente.
	  \item Desarrollar un prototipo capar de recomponer listas de correo \textit{atacando} 
		colecciones de ficheros RDF.
	  \item Abrir la puerta a nuevas aplicaciones.
     	\end{itemize}
  \end{itemize}
}

\subsection{La Web Semántica}
\frame
{
  \frametitle{Introducción a la Web Semántica (I)}

  Tim Berners-Lee expuso en 2001 su visión de lo que sería la Web Semántica:

  \begin{quote}
	\emph{«... \textbf{disponer datos} en la Web \textbf{definidos y enlazados} 
	de forma que puedan ser \textbf{utilizados por las máquinas}, no solamente 
	para visualizarnos, sino también para \textbf{automatizar} tareas, 
	\textbf{integrar} y \textbf{reutilizar} datos entre aplicaciones.»}
  \end{quote}
}
\frame
{
  \frametitle{Introducción a la Web Semántica (II)}

  \begin{columns}
    \begin{column}{0.6\textwidth}
	\includegraphics[width=\textwidth]{images/pila-web-semantica.png}
    \end{column}
    \begin{column}{0.4\textwidth}
      Tecnologías:
      \begin{itemize}
	\item \textbf{RDF} (\textit{Resource Description Framework})
	\item \textbf{OWL} (\textit{Web Ontology Language})
	\item \textbf{SPARQL} (\textit{SPARQL Protocol and RDF Query Language})
      \end{itemize}
    \end{column}
  \end{columns}
}
\frame
{
  \frametitle{RDF}

  \note[item]{	RDF se basa en un modelo de tripletas. El sujeto es un recurso, identificado
		por una URI, que se relaciona mediante un predicado binario con el objeto, que 
		puede ser otra URI o un literal.}
  \note[item]{	Cada tripleta puede verse como un arco, y al juntarse con otros arcos 
		se obtiene un grafo dirigido que describe los recursos y las relaciones 
		entre todos los recursos.}

  \begin{center}
    Modelo de tripletas del tipo \texttt{(sujeto, predicado, objeto)}:
    \includegraphics[width=0.8\textwidth]{images/grafo-rdf.png}
  \end{center}


}
\frame
{
  \frametitle{Ontología}

  Una \textbf{ontología} es FIXME (indio)
}

\section{El proyecto SWAML}

\subsection{Tecnologías implicadas}
\frame
{
  \frametitle{Tecnologías implicadas}

  \begin{itemize}
    \item<2-> \textbf{SIOC}: Semantically-Interlinked Online Communities.
    \item<3-> \textbf{Python}: lenguaje de script orientado a objetos.
    \item<4-> \textbf{RDFLib}: biblioteca de Python para trabajar con RDF.
  \end{itemize}
  FIXME(¿más?)
}

\subsection{SWAML}
\frame
{
  \frametitle{SWAML}

  SWAML se compone de varios componentes:

  FIXME(diagrama)

  \begin{itemize}
   \item SWAML propiamente dicho
   \item Buxon
   \item herramientas complementarias
  \end{itemize}
}
\frame
{
  \frametitle{SWAML Core}
  \begin{columns}
   \begin{column}{0.5\textwidth}
	\includegraphics[width=0.95\textwidth]{images/swaml.png}
   \end{column}
   \begin{column}{0.5\textwidth}
	Un proceso batch que se encarga de traducir un mbox a su representación en RDF:
	\begin{enumerate}
	 \item Parsear el mbox
	 \item Recomponer y verificar la consistencia de todos los datos y relaciones
	 \item Serializar a RDF/XML la lista de correo (\texttt{sioc:Forum})
	\end{enumerate}
   \end{column}
  \end{columns}
}
\frame
{
  \frametitle{Buxon}

  Visor de \texttt{sioc:Forum}'s, desarrollado para demostrar que la representación en RDF de una lista de correo
  permite volver recomponerla sin pérdida de información.

  \begin{center}
	\includegraphics[width=0.7\textwidth]{images/buxon.png}
  \end{center}
}
\frame
{
  \frametitle{Herramientas complementarias}

  Además son necesarias una serie de herramientas adicionales para realizar determinadas tareas:

  \begin{itemize}
   \item \textbf{configWizard}, asistente para realizar configuraciones.
   \item \textbf{FOAF Enricher}, agente encargado de enriquecer la listas de correo con datos 
	 sacados de los FOAF de los suscriptores.
   \item \textbf{KML Exporter}, componente encargado de exportar a KML la información geográfica de los suscriptores.
  \end{itemize}
}

\section{Conclusiones}
\subsection{Impacto}
\frame
{
  FIXME\\
  aportación a SIOC
}
\subsection{Futuras líneas de trabajo}
\frame
{
  \frametitle{Futuro inmediato}

  \note[item]{ Usar GMail como fuente de información. }
  \note[item]{ Bastará disponer de una (o varias) cuentas de GMail suscritas a diferentes
    listas de correo. Después apenas habrá que optimizar el software para
    hacer más eficientes estas exportaciones automáticas. }
  \note[item]{ Esperamos llevar a cabo el experimento en la sproximas semanas. }

  \bigskip
  \begin{columns}
   \begin{column}{0.7\textwidth}
    \includegraphics[width=0.99\textwidth]{images/gmail-swaml.png}
   \end{column}
   \begin{column}{0.4\textwidth}
    Automatizar el experimento aislado ya realizado por John Breslin:
    \begin{enumerate}
     \item Descargar los correos (\texttt{python-libgmail})
     \item Generar un mbox
     \item Usar SWAML para generar su descripción en RDF
    \end{enumerate}
   \end{column}
  \end{columns}
  \bigskip
}
\frame
{
  \frametitle{Futuro a medio plazo}
  \begin{itemize}
   \item Marcado semántico para el cuerpo de los mensajes
   \item API en Python para SIOC
   \item Integración con Mailman
   \item API para DIG
  \end{itemize}
}

\section{Demostración}
\subsection{Demostración}
\frame
{
  \note[item]{Si no desean realizar ningún comentario, procederé a realizar una demostración práctica.}
  \note[item]{Ejemplo de exportación con SWAML (con todas las opciones), enseñar los RDF en plano.}
  \note[item]{Visualizarla con Buxon.}
  \note[item]{¿Algún ejemplo de SPARQL?}
  \note[item]{Terminar enseñando el KML en Google Maps}

  \begin{center}
    \LARGE{\textbf{demostración práctica}}
  \end{center}
}
\subsection{Preguntas}
\frame
{
  \note[item]{Agradecer la atención prestada.}
  \note[item]{Quedar a disposición del tribunal para contestar a las
		preguntas o ampliar cualquiera de los temas expuestos.}

  \begin{center}
    \LARGE{\textbf{SWAML, publicación de listas de correo en web semántica}}\\
    \vspace{1cm}
    \LARGE{Fin}\\
    \vspace{2cm}
    \small{%
	Esta presentación se distribuye bajo los términos de la licencia
	CreativeCommons Reconocimiento-CompartirIgual 2.5
    }
  \end{center}
}

\appendix

\section{Respuestas preparadas}
\frame{
  FIXME: retocarlas según el tribunal
}


\end{document}
